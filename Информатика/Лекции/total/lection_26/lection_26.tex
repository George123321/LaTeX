\chapter{Z и префикс функция строки}
\section{Z--функция строки}
Z--функция строки --- функция от номера символа.
Z--функция --- это массив длинны \texttt{len(S) = N}, \texttt{z[i]} --- длина совпадающего префикса у строки \texttt{S} и \texttt{S[i:]}.

\texttt{z[0]} не определено. Но мы будем считать, что \texttt{z[0] = 0}. 

\begin{center}
"a a a a a"\\
z=[0, 4, 3, 2, 1]\\

"a b a c a b a"\\
z=[0, 0, 1, 0, 3, 0]
\end{center}

Зачем нужна z--функция?
Будем искать строчку \texttt{p=aba} и все ее вхождения в строке \texttt{abacabadabacaba}. Склеим две строки символом, которого точно нет ни там ни там:
\begin{center}
	\texttt{s = "aba\#abacabadabacaba"}.
\end{center}
Т.к. стоит символ \#, длина искомой подстроки не может быть больше 3.
$$
z = [0, 0, 1, 0, 3, 0, 1, 0, 3, 0, 1, 0, 3, 0, 1, 0, 3, 0, 1]
$$
Там, где \texttt{z[i] == len(p)}, т.е. там, где величина Z--функции равна длине подстроки, у нас есть совпадение, т.е. там подстрока содержится в строке. Позиция вхождения: найдена подстрока в строке, \texttt{pos = i-len(p)-1} --- номер вхождения.

Тривиальное вычисление Z--функции (требует $O(N^2)$).
\begin{infa}{Тривиальное вычисление Z--функции}
\num N = \leni(s)
\num z=[0]
\num left = right = 0
\num \fori i \ini \rangei(1, N):
\num \tab x = 0
\num \tab \whilei i + x < N \andi s[x] == s[i+x]:
\num \tab \tab x += 1
\num \tab z[i] = x
\num \tab if i + x - 1> right: \com{Сохраянем z--блок}
\num \tab \tab left, right = i, i + x - 1
\end{infa}
z--блок --- срез строки \texttt{s[i:i+z[i]]}, т.е. это часть строки, совпавшая с подстрокой.

На момент вычисления \texttt{z[i]} существует самый правый отрезок совпадения. Длина этого отрезка равна разнице его правого и левого конца + 1.

\begin{infa}{}
	\ \num N = \leni(s)
	\ \num z=[0]
	\ \num left = right = 0
	\ \num \fori i \ini \rangei(1, N):
	\ \num \tab x = \mini(z[i-left], right - i + 1) \ifi i<=right \elsei 0
	\ \num \tab \whilei i + x < N \andi s[x] == s[i+x]:
	\ \num \tab \tab x += 1
	\ \num \tab z[i] = x
	\ \num \tab \ifi i + x - 1> right: \com{Сохраянем z--блок}
	\num \tab \tab left, right = i, i + x - 1
\end{infa}
Этот алгоритм работает за линейное время.
\section{Префикс--функция строки. Алгоритм Кнута — Морриса — Пратта.}
\subsection{Префикс--функция строки}
Собственным суффиксом строки называется суффикс, не совпадающий со всей строкой, совпадающий с ее префиксом. 

Префикс--функция строки \texttt{$\pi$[i]} --- массив длинной строки, где \texttt{$\pi$[i]} --- длина наибольшего по длине собственного суффикса подстроки (среза) \texttt{s} начиная от начала и до позиции \texttt{i (s[:i+1])}.
\begin{center}
"a a a a a"\\
pi=[0, 1, 2, 3, 4]\\
"a b a c a b a"\\
pi = [0, 0, 1, 0, 1, 2, 3]\\
\end{center}
Заметим, что эта функция всегда растет на единицу.
\begin{infa}{Тривиальный алгоритм}
\num N = \leni(s)
\num pi = [0]*N
\num \fori i \ini \rangei(1, N):
\num \tab \fori k \ini \rangei(i+1):
\num \tab \tab \ifi s[0:k] == s[i-k+1:i+1]:
\num \tab \tab \tab pi[i] = k
\end{infa}
Асимптотика $O(N^3)$.

\begin{infa}{Эффективный алгоритм}
\num \defi prefix(s):
\num \tab n = \leni(s)
\num \tab pi = [0]*n
\num \tab \fori i \ini \rangei(1, n):
\num \tab \tab j = pi[i-1]
\num \tab \tab \whilei j > 0 \andi s[i] != s[j]:
\num \tab \tab \tab j = pi[j-1]
\num \tab \tab \ifi s[i] == s[j]:
\num \tab \tab \tab j += 1
\num \tab \tab pi[i] = j
\num \tab \returni pi
\end{infa}

\subsection{Поиск подстроки в строке}
Эта задача является классическим применением префикс-функции (и, собственно, она и была открыта в связи с этим).

Дан текст t и строка s, требуется найти и вывести позиции всех вхождений строки s в текст t.

Обозначим для удобства через n длину строки s, а через m — длину текста t.

Образуем строку s + \# + t, где символ \# — это разделитель, который не должен нигде более встречаться. Посчитаем для этой строки префикс-функцию. Теперь рассмотрим её значения, кроме первых n+1 (которые, как видно, относятся к строке s и разделителю). По определению, значение $\pi$[i] показывает наидлиннейшую длину подстроки, оканчивающейся в позиции i и совпадающего с префиксом. Но в нашем случае это $\pi$[i] — фактически длина наибольшего блока совпадения со строкой s и оканчивающегося в позиции i. Больше, чем n, эта длина быть не может — за счёт разделителя. А вот равенство $\pi$[i] = n (там, где оно достигается), означает, что в позиции i оканчивается искомое вхождение строки s (только не надо забывать, что все позиции отсчитываются в склеенной строке s+\#+t).

Таким образом, если в какой-то позиции i оказалось $\pi$[i] = n, то в позиции i - (n + 1) - n + 1 = i - 2 n строки t начинается очередное вхождение строки s в строку t.

Как уже упоминалось при описании алгоритма вычисления префикс-функции, если известно, что значения префикс-функции не будут превышать некоторой величины, то достаточно хранить не всю строку и префикс-функцию, а только её начало. В нашем случае это означает, что нужно хранить в памяти лишь строку s + \# и значение префикс-функции на ней, а потом уже считывать по одному символу строку t и пересчитывать текущее значение префикс-функции.

Итак, алгоритм Кнута-Морриса-Пратта решает эту задачу за O(n+m) времени и O(n) памяти.

Подробнее материал лекции изложен на \href{http://e-maxx.ru/algo/prefix_function}{сайте}.