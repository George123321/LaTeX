\chapter{Автоматы}
\section{Машина Тьюринга}
Это абстрактный исполнитель, живущий на бесконечной ленте, в клетках которой находятся буквы $\alpha$ принадлежащие фиксированному алфавиту A. Каретка машины Тьюринга может двигаться по ленте влево или вправо. Каретка может видеть, что нарисовано на ленте (на текущей клетке). Также у нее есть возможность изменять символы на ленте, т.е. она может записать, изменить. А также у нее есть состояние q. При этом это состояние принадлежит множеству допустимых состояний Q. Подмножество состояний --- состояние останова (остановки). Это подмножество конечно. Поведение этой машины детерминированно (оно задается увиденным символом и предыдущим состоянием). Она из исходного состояния переходит в новое состояние, в котором определено: 1) Состояние системы 2) Считанный символ 3) Действие.

Множества Q и A конечные. Возможно очень большие, но конечные. Мощность множества --- число конечных состояний в нем.

Программа, которую мы написали, не хранится нигде. Типа в памяти каретки. Нужно её куда--то записать. Самое простое --- написать не ленте. Тогда каретка, которая едет на двух лентах сразу --- универсальная (программируемая) машина Тьюринга. О скорости работы нет разговора. Есть вопрос о вычислимости алгоритма.
\section{Вычислимость функций (алгоритмов)}
Что по сути такое алгоритм? Есть определенные возможные входные данные. Есть множество значений --- множество возможных результатов. Алгоритм --- своего рода функция, которая переводит множество определения в множество значений. Но невычислимые функции. Вычислимые функции (алгоритмы) --- это алгоритмы. Функции называются вычислимыми, если есть возможность посчитать её через машину Тьюринга. То, что мы называем раличными алгоритмами (все виды сортировок) --- это, с точки зрения вычислимости - один алгоритм. Нам ведь не важен путь (как и не важна скорость). Нам важно, что есть возможность получить результат, не более того. А какие алгоритмы невычислимые? Например, доказано, что нельзя вычислить вычислимость программы. Т.е. невозможно написать программу, которая посчитает, закончится программа или нет.

Исполнители А и В называются алгоритмически эквивалентными, если можно сэмулировать А на В и В на А.

\section{Клеточные автоматы. Игра жизнь Джона Конвая}
Простейшие автоматы --- это клетки, живущие не линии. В клетках --- нули и единицы. Состояние клетки зависит только от самой клетки и от двух ее ближайших соседей. В каждый следующий момент времени клетка меняет свое состояние в зависимости от своего и соседей состояний. При этом для всех клеток алгоритмы одинаковы. Всего есть 256 клеточных автоматов (возможных комбинаций для данных состояний триад клеток). Среди них есть и совсем простые. Интересны несколько из них. Одно --- правило 30, т.к. порождает случайные хаотические структуры. Также есть правила жизни Джона Конвея. Если клетка была жива, то остается живой при 2 или 3 соседях. А если была мертва, то оживает при наличии трех соседей.