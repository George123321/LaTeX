\chapter{Z--функция строки и ее вычисление}
Основной материал лекции взят с \href{http://e-maxx.ru/algo/z_function}{сайта}.
\section{Z--функция}
\subsection{Определение}
Пусть дана строка $s$ длины $n$. Тогда Z--функция от этой строки --- это массив длины $n$, i-ый элемент которого равен наибольшему числу символов, начиная с позиции i, совпадающих с первыми символами строки $s$.

Иными словами, \texttt{z[i]} -- это наибольший общий префикс строки $s$ и её i--го суффикса.

Во избежание неопределённости, мы будем считать строку 0--индексированной --- т.е. первый символ строки имеет индекс 0, а последний --- $n-1$.

Первый элемент Z--функции, \texttt{z[0]}, обычно считают неопределённым. Мы будем считать, что он равен нулю.

\subsection{Реализация}
\subsubsection{Тривиальный алгоритм}
\begin{infa}{Тривиальный алгоритм}
\num \defi z_function_trivial(s):
\num \tab n = \leni(s)
\num \tab z = [0]*n
\num \tab i = 1
\num \tab \whilei i < n:
\num \tab  \tab \whilei i + z[i] < n \andi s[z[i]] == s[i+z[i]]: \com{Пока мы не дошли до\\\phantom{3\ \ \ \ \ \ \ } конца и символ на позиции z[i] равен символу на рассматриваемой позиции + z[i]\\\phantom{3\ \ \ \ \ \ \ } (т.е., это и есть основная проверка)}
\num \tab \tab \tab z[i] += 1
\num \tab \tab i += 1
\num \tab \returni z
\end{infa}
Сложность такого алгоритма $O(N^2)$. 

\subsubsection{Эффективный алгоритм}
Чтобы получить эффективный алгоритм, будем вычислять значения \texttt{z[i]} по очереди — от i=1 до $n-1$, и при этом постараемся при вычислении очередного значения \texttt{z[i]} максимально использовать уже вычисленные значения.
\begin{infa}{Эффективная реализация Z--функции}
\ \num \defi z_function(s):
\ \num \tab z = [0]*\leni(s)
\ \num \tab left = 0  
\ \num \tab right = 0
\ \num \tab x = 0
\ \num \tab \fori i \ini \rangei(1, \leni(s)):
\ \num \tab \tab \ifi i <= right:
\ \num \tab \tab \tab x = \mini(z[i-left], right - i + 1)
\ \num \tab \tab \elsei:
\num \tab \tab \tab x = 0 
\num \tab \tab \whilei i+x < \leni(s) \andi s[x] == s[i+x]:
\num \tab \tab \tab x += 1
\num \tab \tab \ifi i + x - 1  > right:
\num \tab \tab \tab left, right = i, i + x - 1
\num \tab \tab z[i] = x
\num \tab \returni z
\end{infa}
Этот алгоритм выполняется за линейное время.

\subsection{Применение}
Применения Z--функции:
\begin{itemize}
	\item Поиск подстроки в строке.
	\item Поиск количества различных подстрок в строке.
	\item Сжатие строки.
\end{itemize}

Запишем реализацию поиска количества различных подстрок в строке.
\begin{infa}{Поиск количества различных подстрок в строке}
\ \num \defi count_different_substrings(s):
\ \num \tab n = \leni(s)
\ \num \tab start = s[0]
\ \num \tab res = 0
\ \num \tab \fori i \ini \rangei(1, n):
\ \num \tab \tab t = start[::-1]
\ \num \tab \tab k = \leni(t) - \maxi(z_function(t))
\ \num \tab \tab res += k
\ \num \tab \tab start += s[i]
\num \tab \returni res
\end{infa}