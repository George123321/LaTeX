\begin{prim}
Рассмотрим $\varphi:\mathbb{R}^3 \rightarrow \mathbb{R}^3$.

Вопрос: всегда ли существует одномерное инвариантное подпространство?
\end{prim}
Рассмотрим определитель третьего порядка:
$$P(\lambda)=\left| \begin{array}{ccc}
a_{11}- \lambda & a_{12} & a_{13}  \\
a_{21} & a_{22} -\lambda& a_{23}\\
a_{31}&a_{32}&a_{33}-\lambda
\end{array}\right| = -\lambda^3 + \beta_1\lambda^2 +\beta_2\lambda+\beta_3$$

$\begin{aligned}
P(\lambda) &\xrightarrow{\lambda \rightarrow +\infty} -\infty\\
P(\lambda) &\xrightarrow{\lambda \rightarrow -\infty} +\infty
\end{aligned} \Rightarrow P(\lambda)\text{ пересечет ноль и сменит знак} \Rightarrow \exists \lambda_0:P(\lambda_0)=0$

Следовательно, существует вещественное собственное значение $\Rightarrow \exists$ собственный вектор $\Rightarrow \exists$ одномерное инвариантное пространство.

Этот же вывод справедлив для любой нечетной степени характеристического многочлена.
\begin{prim}
Доказать, что характеристический многочлен не зависит от выбора базиса.
\end{prim}
Рассмотрим преобразование $\varphi$ с матрицей $A$,

$A'=S^{-1}AS$, характеристический многочлен $\det(A'-\lambda E)$.

$\det(A'-\lambda E) = \det(S^{-1}AS -\lambda S^{-1}S)= \det(S^{-1}(A-\lambda E)S)=\det S^{-1}\det(A-\lambda E)\det S=\det(SS^{-1})\det(A-\lambda E)=\det(A-\lambda E).$\\
Отсюда следует:
$$
\det(A'-\lambda E)=\det(A-\lambda E)
$$