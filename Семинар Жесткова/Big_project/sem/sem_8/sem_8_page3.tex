Очевидно, собственные значения не меняются при замене базиса.
~\\\
\hrule
~\\\
Рассмотрим подробнее характеристический многочлен.
$$
P(\lambda)=\det (A-\lambda E)=
\begin{vmatrix*}[c]
 a_{11}-\lambda & a_{12} & \cdots & a_{1n}\\
 a_{21} & a_{22}-\lambda & \cdots & a_{2n}\\
 \vdots & \vdots & \ddots & \vdots \\
 a_{n1} & a_{n2} & \cdots & a_{nn}-\lambda\\
\end{vmatrix*}
=
$$
$$
=\underset{\text{только в этом члене есть $\lambda^{n}$ и $\lambda^{n-1}$ }}
{( a_{11}-\lambda)(a_{22}-\lambda)\dots ( a_{nn}-\lambda)} + \dots + \tilde P(0)=
$$
$$
=(-1)^{n}\lambda^{n}-(-1)^{n}( a_{11}+a_{22}+\dots+a_{nn})\lambda^{n-1}+\dots+\det A
$$
Вспомним, что для квадратного уравнения вида 
$$ax^{2}+bx+c=0,\quad a\neq 0$$
справедлива теорема Виета, которую мы знаем еще из школы:
$$\left\{
\begin{aligned}
x_{1}+x_{2}=-b/a\\
x_{1}x_{2}=c/a\\
\end{aligned}
\right.$$

\textbf{Обобщенная теорема Виета:}\\
Произведение корней: $(-1)^{n}\cdot \frac{\{ \text{свободный член}\}}{\{ \text{первый коэффициент}\}}$\\
Сумма корней: $-\frac{\{ \text{второй коэффициент}\}}{\{ \text{первый коэффициент}\}}$\\
$\lambda_{1}\lambda_{2}\dots \lambda_{n}=(-1)^{n}\cdot \frac{\det A}{(-1)^{n}}=\det A$\\
$\lambda_{1}+ \dots + \lambda_{n}=(a_{11}+a_{22}+\dots+a_{nn})=\tr A$ (т.е. след матрицы $A$).\\
Т.о. оказывается , что $\det A$ и $\tr A$ не зависят от выбора базиса.
\begin{prim}
Пусть A --- матрица вращения $\mathbb{R}^{3}$. Найти угол вращения.
\end{prim}
Выберем ортонормированный базис так, что поворот вокруг \textbf{e$_{3}$}.\\
В этом базисе: