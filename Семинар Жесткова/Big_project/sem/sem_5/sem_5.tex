\part{Линейные отображения. Часть 1.}
\section{Определение линейного отображения}
\begin{definition}
	Пусть заданы $L$ и $\overline{L}$ --- два линейных пространства. \textsf{Отображение} $\phi$ из $L$ в $\overline{L}$ --- правило, по которому \underline{каждому} вектору из $L$ ставится в соответствие \underline{единственный} вектор из $\overline{L}$.\\
	Обозначение: $\phi: L\rightarrow\overline{L}$.
\end{definition}

\begin{definition}
	\textsf{Сюрьекция} --- отображение, при котором каждый элемент из $\overline{L}$ является образом вектора из $L$.
\end{definition}

\begin{figure}[h!]
	\centering
	\def\svgwidth{5cm} % если надо изменить размер
	\input{sem/sem_5/surekt.pdf_tex}
	\caption{Сюръективная функция}
	\label{...}
\end{figure}

\begin{definition}
	\textsf{Инъекция} --- отображение, при котором каждый образ из $\overline{L}$ имеет единственный прообраз в $L$.
\end{definition}

\begin{figure}[h!]
	\centering
	\def\svgwidth{5cm} % если надо изменить размер
	\input{sem/sem_5/inekt.pdf_tex}
	\caption{Инъективная функция}
	\label{...}
\end{figure}

\begin{definition}
	Сюрьекция + инъекция = \textsf{биекция} --- это отображение, которое является одновременно и сюръективным, и инъективным (взаимно--однозначное соответствие).
\end{definition}

\begin{definition}
	Если в результате отображения $L=\overline{L}$, то такое отображение называется \textsf{преобразованием}.
\end{definition}

\begin{definition}
	Отображение $\pi$ называется \textsf{линейным}, если выполнено:
	$$
	\left\{
	\begin{array}{l}
	\phi(x+y)=\phi(x)+\phi(y)\\
	\phi(\alpha x)=\alpha\phi(x).\\
	\end{array}\right.
	$$
\end{definition}

Примеры:
\begin{itemize}
	\item аффинные преобразования плоскости
	\item геометрия 3D (т.е. преобразования в трёхмерном пространстве: поворот, симметрия и т.д.)
	\item присвоение координат в линейном пространстве (пространство многочленов $\rightarrow$ пространство столбцов т.е. координатный изоморфизм)
	\item умножение на матрицу, транспонирование
	\item дифференцирование, интегрирование (только для определённого интеграла, чтобы избежать появление константы)
\end{itemize}
Очевидно, что $\phi(0)=0$.

Рассмотрим ЛЗ систему векторов $a_1,\dots,a_n$.
$$
\alpha_1 a_1+\dots+\alpha_n a_n=0.
$$
Подействуем преобразованием $\phi$:
$$
\alpha_1 \phi(a_1)+\dots+\alpha_n \phi(a_n)=0.
$$
Отсюда видно, что \textsf{система образов} --- тоже ЛЗ система векторов с теми же коэффициентами. Но для ЛНЗ системы векторов данное утверждение не верно (контрпример: нуль--преобразование).

\begin{definition}
	\textsf{Образ} $\phi: \im \phi:\{\phi(x)\in \overline{L}:x\in L\}$ --- множество всех образов из $L$ в $\overline{L}$.
\end{definition}

\begin{definition}
	\textsf{Ядро} $\phi: \ker{\phi}: \{x\in L:\phi(x)=0\}$ --- множество векторов из $L$, которые переходят в 0.
\end{definition}

\begin{definition}
	\textsf{Ранг} $\phi: r=\dim(\im{\phi})$ --- размерность образа.
\end{definition}

\begin{wrapfigure}{r}{0.27\linewidth}
	\def\svgwidth{5cm} % если надо изменить размер
	\input{sem/sem_5/prim1.pdf_tex}
	\caption{К примеру 1а}
	\label{...}
\end{wrapfigure}

\begin{prim}
	Работаем в $\mathbb{R}^3$, ОНБ, \textbf{a}$\neq$\textbf{0}, \textbf{n}$\neq$\textbf{0}.\\
	Найти \phi, если \phi --- ортогональная проекция на:\\
	а) $L_1$: \textbf{[r, a] = 0}\\
	б) $L_2$: \textbf{(r, n)} = 0
\end{prim}



\noindent а) $\phi(\textbf{x})=\cfrac{\textbf{(x,\,a)}}{\abs{\textbf{a}}^2}\,\textbf{a}$ --- формула для проекции вектора на прямую (из аналит. геометрии).\\
$\ker \phi : \textbf{(r,\,a)}=0$ --- плоскость, ортогональная вектору $\textbf{a}$.\\
$\im \phi: \textbf{[r,\,a]=0}$.\\
$r = 1$ (прямая).

\begin{wrapfigure}{r}{0.27\linewidth}
	\def\svgwidth{5cm} % если надо изменить размер
	\input{sem/sem_5/prim1b.pdf_tex}
	\caption{К примеру 1б}
	\label{...}
	\vspace{-10cm}
\end{wrapfigure}

\noindent б) $\phi(\textbf{x})=\textbf{x}-\textbf{p}$\\
$\phi(\textbf{x})=\textbf{x}-\cfrac{\textbf{(\textbf{x},\,\textbf{n})}}{\abs{\textbf{n}}^2}\,\textbf{n}$.\\
$\ker \phi: \textbf{[r,\,n]=0}.$\\
$\im \phi: \textbf{(r,\,n)}=0$ (плоскость).\\
$r=2$ (плоскость).

\section{Матрица линейного отображения}
\noindent$\phi: L\rightarrow\overline{L}, \dim L=n<\infty, \dim \overline{L}=m<\infty$\\
Базисы в $L: \textbf{e} (e_1,\dots,e_n)$, в $\overline{L}: \textbf{f} (f_1,\dots,f_m)$, $a \in L$. $\phi(a) \in \overline{L}$.\\
Пусть $a$ имеет в $L$ координатный столбец $\textbf{x}$, а $\phi(a)$ в $\overline{L}$ координатный столбец $\textbf{y}$. Построим такую матрицу $A$: $\underset{m\times n} A\cdot \underset{n\times 1}{\text{\textbf{x}}}=\underset{m\times 1}{\text{\textbf{y}}}$.\\
Пусть $\textbf{x}=e_1: (1\ 0\ \cdots\ 0)^{\text{T}}$.\\
$y=\underset{\text{в базисе $f$}}{\phi(e_1)}=Ae_1=A\cdot (1\ 0\ \cdots\ 0)^{\text{T}}=a_1$ --- первый столбец из A. Аналогично поступим с вторым, третьим и т.д. столбцами. Тогда матрица линейного отображения A имеет вид:
$$A=
\left.\left(\hspace{1mm}
\begin{array}{|c|c|c|}
\cline{1-1} \cline{3-3}
\multirow{4}{*}{$\phi(e_1)$} & \multirow{4}{*}{$\cdots$} & \multirow{4}{*}{$\phi(e_n)$} \\
&                           &                              \\
&                           &                              \\
&                           &                              \\ \cline{1-1} \cline{3-3} 
\end{array}\hspace{1mm}\right) \right\}m
$$
В данном случае, столбцы матрицы --- это координатные столбцы $\phi(e_i)$ в базисе $f$ т.к.:
$$
a=\alpha_1 e_1+\dots+\alpha_n e_n
$$
Подействуем на него преобразованием $\phi$:
$$
\phi(a)=\alpha_1 \phi(e_1)+\dots+\alpha_n \phi(e_n)
$$

В примерах 2--5 вопрос следующий: найти $A$, если задано $\phi$ --- преобразование $\mathbb{R}^3$, в ОНБ.
\begin{prim}
	$\phi$ --- поворот вокруг $\textbf{$e_3$}$ на угол $\frac{\pi}{2}$.
\end{prim}

\begin{wrapfigure}{r}{0.27\linewidth}
	\def\svgwidth{3cm} % если надо изменить размер
	\input{sem/sem_5/prim2.pdf_tex}
	\caption{К примеру 2}
	\label{...}
	\vspace{-9cm}
\end{wrapfigure}

\noindent$\underset{\text{e$_1$,e$_2$,e$_3$}}{L}\rightarrow\underset{\text{e$_1$,e$_2$,e$_3$}}{\overline{L}}$\footnote{Всегда в решении задачи обязательно нужно выбрать базисы.}.\\
$\phi(\textbf{e$_1$})=\textbf{e$_2$}: \begin{psm}
0\\1\\0\\
\end{psm}\\
\phi(\textbf{e$_2$})=-\textbf{e$_1$}: \begin{psm}
-1\\0\\0\\
\end{psm}\\
\phi(\textbf{\text{e$_3$}})=\textbf{e$_3$}:\begin{psm}
0\\0\\1\\
\end{psm}$\\
$$A=\begin{pmatrixr}
0&-1&0\\
1&0&0\\
0&0&1\\	
\end{pmatrixr}$$
Для построения матрицы $A$ нам необходимо и достаточно образов преобразования.
\begin{prim}
	$\phi$ --- ортогональное проецирование на $L: x=y=z$.
\end{prim}

\begin{wrapfigure}{r}{0.35\linewidth}
	\def\svgwidth{6cm} % если надо изменить размер
	\input{sem/sem_5/prim3.pdf_tex}
	\caption{К примеру 3}
	\label{...}
	\vspace{-9cm}
\end{wrapfigure}

\noindent$\underset{\text{e$_1$,e$_2$,e$_3$}}{L}\rightarrow\underset{\text{e$_1$,e$_2$,e$_3$}}{\overline{L}}$\\
$\phi(\textbf{x})=\cfrac{\textbf{(x,\,a)}}{\abs{\textbf{a}}^2}\textbf{a}$\\
<<Прогоним>> через эту формулу все базисные векторы:\\
$\left.
\begin{aligned}
\phi(\textbf{e$_1$})=\frac{1}{3}\begin{psm}
1\\1\\1
\end{psm}\\
\phi(\textbf{e$_2$})=\frac{1}{3}\begin{psm}
1\\1\\1
\end{psm}\\
\phi(\textbf{e$_3$})=\frac{1}{3}\begin{psm}
1\\1\\1
\end{psm}
\end{aligned}
\right\}\Rightarrow A = \cfrac{1}{3}\begin{pmatrixr}
1&1&1\\1&1&1\\1&1&1\\
\end{pmatrixr}
$
\\
\begin{prim}
	$\varphi$ - отражение относительно $\alpha$: $2x-2y+z=0$
\end{prim}\\

\begin{wrapfigure}{r}{0.35\linewidth}
	\vspace{-2.5cm}
	\def\svgwidth{6cm} % если надо изменить размер
	\input{sem/sem_5/prim4.pdf_tex}
	\caption{К примеру 4}
	\label{...}
	\vspace{-9cm}
\end{wrapfigure}

$\underset{\text{e$_1$,e$_2$,e$_3$}}{L}\rightarrow\underset{\text{e$_1$,e$_2$,e$_3$}}{\overline{L}}$\\
$\textbf{n} (2;-2;1)$\\
$|\textbf{n}|^2=9$\\
$\varphi(\textbf{p})=\textbf{p}-2\cfrac{\textbf{(p,\,a)}}{|\textbf{a}|^2}\textbf{n}$\\
$\varphi(\textbf{e$_1$})=
\left(
\begin{smallmatrix*}[r]
1\\ 0\\ 0\\
\end{smallmatrix*}
\right)
-2 \cdot \frac{2}{9} 
\left(
\begin{smallmatrix*}[r]
2\\ -2\\ 1\\
\end{smallmatrix*}
\right)$
;
$\varphi(\textbf{e$_1$})=\frac{1}{9} 
\left(
\begin{smallmatrix*}[r]
1\\ 8\\ -4\\
\end{smallmatrix*}
\right)$;\\
$\varphi(\textbf{e$_2$})=\frac{1}{9} 
\left(
\begin{smallmatrix*}[r]
8\\ -7\\ 4\\
\end{smallmatrix*}
\right)$;\\
$\varphi(\textbf{e$_3$})=\frac{1}{9} 
\left(
\begin{smallmatrix*}[r]
-4\\ 4\\ 1\\
\end{smallmatrix*}
\right)$;\\
$$
A=\frac{1}{9} 
\begin{pmatrix*}[r]
1 & 8 & -4\\
8 & -7 & 4\\
-4 & 4 & 1\\
\end{pmatrix*}
$$\\
\begin{prim}
	$L_1: x=0$\\
	$L_2: 2x=2y=-z$\\
\end{prim}\\

\begin{wrapfigure}{r}{0.4\linewidth}
	\vspace{-2.5cm}
	\def\svgwidth{8cm} % если надо изменить размер
	\input{sem/sem_5/prim5.pdf_tex}
	\caption{К примеру 5}
	\label{...}
	\vspace{-9cm}
\end{wrapfigure}

$\underset{\text{e$_1$,e$_2$,e$_3$}}{L}\rightarrow\underset{\text{e$_1$,e$_2$,e$_3$}}{\overline{L}}$\\

$\varphi$ --- проецирование на $L_1 || L_2$\\
$L_1: \langle
\begin{pmatrix*}[r]
0\\
1\\
0\\
\end{pmatrix*}
,
\begin{pmatrix*}[r]
0\\
0\\
1\\
\end{pmatrix*}
\rangle
$,
$L_2: \langle
\begin{pmatrix*}[r]
\frac{1}{2} \\
\frac{1}{2} \\
-1\\
\end{pmatrix*}
\rangle$\\
$ \textbf{p}=
\underbrace{
	\alpha
	\left(
	\begin{smallmatrix*}[r]
	0\\ 1\\ 0\\
	\end{smallmatrix*}
	\right)
	+\beta
	\left(
	\begin{smallmatrix*}[r]
	0\\ 0\\ 1\\
	\end{smallmatrix*}
	\right)
}_{\in L_1}
+
\underbrace{
	\gamma
	\left(
	\begin{smallmatrix*}[r]
	\frac{1}{2} \\
	\frac{1}{2} \\
	-1\\
	\end{smallmatrix*}
	\right)
}_{\in L_2}\\
\textbf{e$_1$}: 
\begin{psm}
1\\ 0\\ 0\\
\end{psm}
=
\underbrace{
	\alpha
	\left(
	\begin{smallmatrix*}[r]
	0\\ 1\\ 0\\
	\end{smallmatrix*}
	\right)
	+\beta
	\left(
	\begin{smallmatrix*}[r]
	0\\ 0\\ 1\\
	\end{smallmatrix*}
	\right)
}_{\varphi(\textbf{e$_1$})}
+\gamma
\left(
\begin{smallmatrix*}[r]
\frac{1}{2} \\
\frac{1}{2} \\
-1\\
\end{smallmatrix*}
\right)
\Rightarrow
\gamma=2
\Rightarrow
\varphi(\textbf{e$_1$})=
\left(
\begin{smallmatrix*}[r]
0\\ -1\\ 2\\
\end{smallmatrix*}
\right)
$\\
$\textbf{e$_2$}: \varphi(\textbf{e$_2$})=\textbf{e$_2$};\\
\textbf{e$_3$}: \varphi(\textbf{e$_3$})=\textbf{e$_3$}.  $\\
Ответ:$
\begin{pmatrix*}[r]
0 & 0 & 0\\
-1 & 1 & 0\\
2 & 0 & 1\\
\end{pmatrix*}
$\\
Понимая пример 5 как отображение, можно заметить, что $\dim(\im\varphi)=2$,  а значит, для описания всевозможных результатов в $\bar L$ можно было выбрать базис из двух векторов.\\
$ f= \left\{ \left(
\begin{smallmatrix*}[r]
0\\ 1\\ 0\\
\end{smallmatrix*}
\right) 
, 
\left(
\begin{smallmatrix*}[r]
0\\ 0\\ 1\\
\end{smallmatrix*}
\right) 
\right\}$\\ 
$\varphi (\textbf{e$_1$})=\left(
\begin{smallmatrix*}[r]
-1\\ 2\\
\end{smallmatrix*}
\right) $
,
$\varphi (\textbf{e$_2$})=\left(
\begin{smallmatrix*}[r]
1\\ 0\\
\end{smallmatrix*}
\right) $
,
$\varphi (\textbf{e$_3$})=\left(
\begin{smallmatrix*}[r]
0\\ 1\\
\end{smallmatrix*}
\right) $;\\
$$A=
\begin{pmatrixr}
-1 & 1 & 0\\
2 & 0 & 1\\
\end{pmatrixr}
$$
\begin{prim}
	$ \varphi : \underset{\textbf{e$_1$}, \textbf{e$_2$}, \textbf{e$_3$}}L \Rightarrow \underset{\textbf{f$_1$}, \textbf{f$_2$}, \textbf{f$_3$}, \textbf{f$_4$}}{\bar L} $\\
	
	$L: \langle\underset{=\text{\textbf{$e_1$}}}1, \underset{=\text{\textbf{$e_2$}}}x, \underset{=\text{\textbf{$e_3$}}}{x^2}\rangle$ , 
	$L: \langle\underset{=\text{\textbf{$f_1$}}}1, \underset{=\text{\textbf{$f_2$}}}x, \underset{=\text{\textbf{$f_3$}}}{x^2}, \underset{=\text{\textbf{$f_4$}}}{x^3}\rangle$ \\
\end{prim}
$\varphi (x)=\int\limits^x_0 f(t)dt$\\
$$
\left.
\begin{aligned}
\varphi (e_1)=\int\limits^x_0 1 dt=x
\left(
\begin{smallmatrix*}[r]
0\\ 1\\ 0\\ 0\\
\end{smallmatrix*}
\right) \\
\varphi (e_2)=\int\limits^x_0 t dt=\frac{x^2}{2}
\left(
\begin{smallmatrix*}[r]
0\\ 0\\ \frac{1}{2}\\ 0\\
\end{smallmatrix*}
\right) \\
\varphi (e_3)=\int\limits^x_0 t^2 dt=\frac{x^3}{3}
\left(
\begin{smallmatrix*}[r]
0\\ 0\\  0\\ \frac{1}{3}\\
\end{smallmatrix*}
\right) 
\end{aligned}
\right\}
A=
\begin{pmatrix*}[r]
0 & 0 & 0\\
1 & 0 & 0\\
0 & \frac{1}{2} & 0\\
0 & 0 & \frac{1}{3}\\
\end{pmatrix*}
$$




