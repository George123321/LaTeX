\begin{prim}
	$
	\text{(Здесь Г}=E \text{)} \vspace{0.2cm}\\
	U\!:~
\begin{matrix}
\overset{\langle
	\begin{pmatrix*}[r]
	1\\ -1\\ 1\\ 0 
	\end{pmatrix*},
	\begin{pmatrix*}[r]
	2\\ -1\\ 0\\ 1
	\end{pmatrix*}
	\rangle}{
	\begin{matrix*}[r]
	~\textbf{a}\phantom{md}&\textbf{b}
	\end{matrix*}}
\end{matrix}
	,
	~~~~~ X = \begin{pmatrix}
	1\\0\\2\\-2
	\end{pmatrix}
	~~~~ \text{Пр}_U^x=~?
	$
\end{prim}
\textbf{\underline{Первый способ:}}\vspace{0.2cm}\\
$
\underbrace {\alpha a + \beta b}_
{\substack{
		~\in~\textbf{\small a} }}
\underbrace {+~c}_
{\substack{
		~\in~\textbf{\small b} }}
$, причем $c\perp a,~c\perp b$.\vspace{0.2cm} \\
Домножим это выражение скалярно на $a$~и на $b$~и составим систему:
$$
\left\{ 
\begin{aligned}
(x, a) = \alpha \abs{a}^2 + \beta (a, b) + 0\\
(x, b) = \alpha (a, b) + \beta \abs{b}^2 + 0
\end{aligned}
\right. \Leftrightarrow 
\left\{
\begin{aligned}
3 = 3\alpha + 3\beta\\
0 = 3\alpha + 6\beta
\end{aligned}
\right.
$$
Отсюда $\alpha=2, \beta = -1$.~Искомая проекция равна:\\
$\text{Пр}_U^x=\alpha a + \beta b = 2a - b = \begin{pmatrix}
0\\ -1\\ 2\\ 1
\end{pmatrix}
$
\\
\textbf{\underline{Второй способ:}}\vspace{0.2cm}\\
Как было показано выше, в ОНБ проекция вектора равна сумме проекций
на каждый из базисных векторов. Однако в случае произвольного базиса это не так:
$$
\text{Пр}_U^x \neq \text{Пр}_a^x + \text{Пр}_b^x \text{ --- не работает, если } a \not\perp b ~~!
$$
Было бы здорово, если бы в $U$ был базис $\{a', b'\}$ такой, что $a'\perp b'$, тогда соотношение будет работать. Для этого \textit{ортогонализируем} базис:\\
\hspace*{0.2cm}
$
a' = a
$\\
\hspace*{0.2cm}
$
b' = b - \text{Пр}_a^b = b - \frac{(b,a)}{\abs{a}^2} a = 
\begin{pmatrix}
1\\ 0\\ -1\\ 1
\end{pmatrix}
\\
\text{Итак, в } U \text{ мы нашли новый базис}\!:~ \{a', b'\} = \left\{
\begin{pmatrix}
1\\ -1\\ 1\\ 0
\end{pmatrix},
\begin{pmatrix}
1\\ 0\\ -1\\ 1
\end{pmatrix}
\right\}\\
\text{Здесь уже можем пользоваться приведенным соотношением для нахождения проекции:}\\
\text{Пр}_U^x = \text{Пр}_{a'}^x + \text{Пр}_{b'}^x = \frac{(x, a')}{\abs{a'}^2} a' + \frac{(x, b')}{\abs{b'}^2} b' = 
\begin{pmatrix}
0\\ -1\\ 2\\ 1
\end{pmatrix}
$
\vspace{0.7cm}
\\
\textit{Замечание} Хотя мы и нашли новый базис, координаты векторов $x$, $a$, $b$ все ещё выражены в старом базисе, а поэтому и скалярное произведение мы считаем, используя матрицу Грама в \underline{старом} базисе.