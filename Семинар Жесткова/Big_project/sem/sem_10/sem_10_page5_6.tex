\begin{definition}
Ортонормированный базис --- базис, в котором
$$
\textbf{(e$_i$, e$_j$)}= % скалярное произведение
\left\{
\begin{aligned}
0, i\ne j\\
1, i=j.\\
\end{aligned}
\right.$$
Здесь $\Gamma=E$, $\textbf{(x, y)}=\bm \xi^{T} \bm \eta$.
\end{definition}

\begin{definition}
Рассмотрим подпространство $U\in E$.
Тогда $U^{\perp}$ называется ортогональным дополнением подпространства $U$, если
$$
U^{\perp}: \{\textbf{y}: \textbf{y}\perp U\},\text{ т.е. }\textbf{y}\perp \textbf{x},  \forall\textbf{x}\in U, U\oplus U^{\perp}= \mathcal{E}$$%ВЕКТОРЫ!!!
\end{definition}
\begin{prim}
Дано $U$. Найти $U^{\perp}$ в $\mathcal{E}^3$. % E^3 - непонятно, что 
\end{prim}
$ \Gamma=E$; 
$U=\langle \left(
\begin{smallmatrix*}[r]
1\\ -5\\ 1\\ 
\end{smallmatrix*}
\right) ,
\left(
\begin{smallmatrix*}[r]
-1\\ 1\\ 1\\ 
\end{smallmatrix*}
\right) \rangle $;
\:\:
$U^{\perp}=\left(
\begin{smallmatrix*}[r]
y_1\\ y_2\\ y_3\\ 
\end{smallmatrix*}
\right)$;

$\left\{
\begin{aligned}
y_1-5y_2+y_3=0;\\
-y_1+y_2+y_3=0;\\
\end{aligned}
\right.$ \:\:\:
$\begin{pmatrix*}[r]
1 & -5 & 1 & \vrule & 0\\
-1 & 1 & 1 & \vrule & 0\\
\end{pmatrix*}
\thicksim
\begin{pmatrix*}[r]
1 & -5 & 1 & \vrule & 0\\
0 & -4 & 2 & \vrule & 0\\
\end{pmatrix*}
\thicksim \\
\thicksim
\begin{pmatrix*}[r]
1 & 0 & -3/2 & \vrule & 0\\
0 & 1 & -1/2 & \vrule & 0\\
\end{pmatrix*}
\Rightarrow
U^{\perp}=
\langle \left(
\begin{smallmatrix*}[r]
3\\ 1\\ 2\\ 
\end{smallmatrix*}
\right) \rangle $\\

Почему $U^{\perp}$ ортогональное дополнение? Дело в том, что его сумма с $U$ дает нам все евклидово пространство.\\
$$\underset{k}U\oplus \underset{n-k}U^{\perp}= \underset{n}{\mathcal{E}}$$