\section{Критерий Сильвестра}
\subsection{Положительно и отрицательно определенные функции}

Квадратичная функция называется \textbf{положительно определенной}, если для любого вектора $x\neq0$, верно: $k(x)>0.$ Например, функция $\mathbf{k}(x)=x^2_1+4x^2_2.$\\

Квадратичная функция называется \textbf{отрицательно определенной}, если для любого вектора $x\neq0$, верно: $k(x)<0.$ Например, функция $\mathbf{k}(x)=-2x^2_1-x^2_2.$\\


Важной является следующая задача: определить, является ли квадратичная функция положительно определенной. Мы уже можем дать ответ на этот вопрос. Для этого можно привести функцию к каноническому виду, и в случае если на диагонали находятся только <<$+1$>>, дать утвердительный ответ, иначе дать отрицательный ответ. Однако, оказывается, что для того, чтобы выяснить положительную определенность функции необязательно приводить ее к каноническому виду. Ответ на поставленный вопрос дает критерий Сильвестра.

\subsection{Критерий Сильвестра}
\begin{theorem}[Критерий Сильвестра]
Для положительной определенности квадратичной функции необходимо и достаточно, чтобы миноры ее матрицы удовлетворяли неравенствам:$$\Delta_k=\begin{vmatrix*}[r]
\beta_{11} & \cdots & \beta_{1k}\\
\hdotsfor{3}\\
\beta_{k1} & \cdots & \beta_{kk}
\end{vmatrix*}>0,\hspace{15pt} k=1,\ldots, n.$$

Миноры в левой части называются \textit{главными минорами} матрицы.
\end{theorem}
\begin{proof}
\emph{Центральный тезис}: при методе элементарных преобразований главные миноры в процессе не меняются в силу свойств детерминанта.

\textit{Необходимость:} в диагональном виде все диагональные элементы положительны, поэтому в исходном виде $M_k>0$.

\textit{Достаточность:} Докажем по индукции. Для первого элемента: $ M_1>0 \Rightarrow \beta_{11}=\epsilon_1>0 $. Тогда на $k$-ом шаге:
$$
B = \left(
\begin{array}{ccc|c}
\epsilon_1 & \dots & 0 &\multirow{3}{0.4cm}{$O$}\\
&\ddots& &\\
0&\hdots&\epsilon_k& \\
\hline
\multicolumn{3}{c|}{O}& C_k\\
\end{array}\right)
$$
В таком виде $ \epsilon_{k+1}=\cfrac{M_{k+1}}{M_k} > 0 $, т.к. $ M_{k+1}>0 $ и $ M_k>0 $.
\end{proof}


\begin{prim}
Является ли квадратичная функция $\textbf{k}(x)=2x^2_1-4x_1x_2+5x^2_2$ положительно определенной?
\end{prim}
Матрица квадратичной функция: $B=\begin{pmatrixr}
2&-2\\
-2&5
\end{pmatrixr}.$

Рассмотрим главные миноры: $\Delta_1=|2|=2>0$, $\Delta_2=\begin{vmatrix*}[r]
2&-2\\
-2&5
\end{vmatrix*}=10-4=6>0.$\\

Все главные миноры положительны, таким образом получили

\underline{Ответ:} да, является положительно определенной.\\

\begin{prim}
Дана квадратичная функция: $\mathbf{k}(x)=x^2_1+\lambda x^2_2+4x^2_3-2x_1x_2+2x_1x_3$. При каком $\lambda$ функция $\mathbf{k}(x)$ положительно определена?
\end{prim}
Матрица квадратичной функции: $B=\begin{pmatrixr}
1&-1& 1\\
-1& \lambda & 0\\
1 & 0 & 4
\end{pmatrixr}.$

Согласно критерию Сильвестра, для положительной определенности квадратичной функции необходимо и достаточно, чтобы ее главные миноры были положительны. Рассмотрим их:

$\Delta_1=|1|=1>0$,\hspace{14pt}$\Delta_2=\begin{vmatrix*}[r]
1&-1\\
-1&\lambda
\end{vmatrix*}=\lambda-1>0$, \hspace{14pt}$\Delta_3=\begin{vmatrix*}[r]
1&-1& 1\\
-1& \lambda & 0\\
1 & 0 & 4
\end{vmatrix*}=3\lambda-4>0.$ \\

Получили систему из двух условий: $\begin{cases}\lambda-1>0,\\
3\lambda-4>0
\end{cases}\Leftrightarrow \lambda>\dfrac{4}{3}.$

\underline{Ответ:} $\lambda>\dfrac{4}{3}.$\\

\begin{prim}
При каких $\alpha$ квадратичная форма $\textbf{k}(x) = 2x_1^2+x_2^2+3x_3^2+2\alpha x_1x_2-2x_1x_3$ положительно определена?
\end{prim}
Матрица квадратичной функции: $B=\begin{pmatrixr}
2&\alpha& -1\\
\alpha& 1 & 0\\
-1 & 0 & 3
\end{pmatrixr}.$

Согласно критерию Сильвестра, для положительной определенности квадратичной функции необходимо и достаточно, чтобы ее главные миноры были положительны. Рассмотрим их:

$\Delta_1=|2|=1>0$,\hspace{14pt}$\Delta_2=\begin{vmatrix*}[r]
2&\alpha\\
\alpha&1\\
\end{vmatrix*}=2-\alpha^2>0$, \hspace{14pt}$\Delta_3=\begin{vmatrix*}[r]
2&\alpha& -1\\
\alpha& 1 & 0\\
-1 & 0 & 3
\end{vmatrix*}=5-3\alpha^2>0.$

Получили систему из двух условий: $\begin{cases}2-\alpha^2>0,\\
5-3\alpha^2>0
\end{cases}\Leftrightarrow \alpha \in \left(-\sqrt{\cfrac{5}{3}}, \sqrt{\cfrac{5}{3}}\right).$

\begin{prim}
Доказать, что для отрицательной определенности квадратичной функции необходимо и достаточно, чтобы знаки главных миноров ее матрицы чередовались, начиная со знака <<$-$>>.
\end{prim}
Рассмотрим квадратичную функцию $\textbf{k}(x)$ с матрицей $B$. Пусть она отрицательно определена. Тогда функция $-\textbf{k}(x)$ с матрицей $-B$ определена положительно. Поэтому критерием (необходимым и достаточным) отрицательной определенности функции $\mathbf{k}(x)$ является положительность всех главных миноров матрицы:$$-B=\begin{pmatrixr}
-\beta_{11} & -\beta_{12} & \cdots & -\beta_{1n}\\
-\beta_{21} & -\beta_{22} & \cdots & -\beta_{2n}\\
 \hdotsfor{4} \\
-\beta_{n1} & -\beta_{n2} & \cdots & -\beta_{nn}
\end{pmatrixr}.$$

Иначе говоря:

$$\Delta_1=-\beta_{11}>0,\hspace{10pt} \Delta_2=\begin{vmatrix*}[r]
-\beta_{11} & -\beta_{12} \\
-\beta_{21} & -\beta_{22}
\end{vmatrix*}>0, \hspace{10pt}\Delta_3=\begin{vmatrix*}[r]
-\beta_{11} & -\beta_{12} & -\beta_{13}\\
-\beta_{21} & -\beta_{22} & -\beta_{23}\\
-\beta_{31} & -\beta_{32} & -\beta_{33}\\
\end{vmatrix*}>0,\hspace{10pt} \ldots$$

Опираясь на свойства детерминанта (вынесем минус из каждой строки, всего $k$ раз, где $k$ -- порядок минора), перепишем последнее:

$$\Delta_1=\beta_{11}<0,\hspace{10pt} \Delta_2=\begin{vmatrix*}[r]
\beta_{11} & \beta_{12} \\
\beta_{21} & \beta_{22}
\end{vmatrix*}>0, \hspace{10pt}\Delta_3=\begin{vmatrix*}[r]
\beta_{11} & \beta_{12} & \beta_{13}\\
\beta_{21} & \beta_{22} & \beta_{23}\\
\beta_{31} & \beta_{32} & \beta_{33}\\
\end{vmatrix*}<0, \hspace{10pt}\ldots $$

Таким образом, доказали важное следствие критерия Сильвестра: для отрицательной определенности квадратичной функции необходимо и достаточно, чтобы знаки главных миноров ее матрицы чередовались, начиная со знака <<$-$>>.\\

С целью не забыть, с какого знака начинается чередование, полезно помнить, квадратичная функция с матрицей $E$ (где $E$ - единичная матрица) положительна определена, а квадратичная функция с матрицей $-E$ отрицательно определена.