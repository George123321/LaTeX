\part{Системы линейных уравнений.}
\section{Запись систем линейных уравнений}
Способы записи систем линейных уравнений:
$$(*)\Leftrightarrow
\begin{cases} 
a_{11}x_1+...+a_{1n}x_n=b_1\\
a_{21}x_1+...+a_{2n}x_n=b_2\\
\cdots\cdots\cdots\cdots\cdots\cdots\cdots\cdots\\
a_{m1}x_1+...+a_{mn}x_n=b_m\\
\end{cases}
$$
Можно записать расширенную матрицу:
$$(*)\Leftrightarrow
\left( \begin{array}{cccc|c}
a_{11} & \hdotsfor{2} & a_{1n} & b_1\\
a_{21} & \hdotsfor{2} & a_{2n} & b_2\\
\hdotsfor{4} & \hdotsfor{1}\\
a_{m1} & \hdotsfor{2} & a_{mn} & b_m\\
\end{array} \right) \text{или}\  (A|b)
$$

Расширенная матрица выдерживает элементарные преобразования строк и перестановку столбцов (\textbf{аккуратно}, т.к. нужно соблюдать нумерацию столбцов).
$$(*)\Leftrightarrow
x_1\begin{pmatrix}
a_{11}\\
a_{21}\\
\vdots\\
a_{m1}
\end{pmatrix}
+\dots+
x_n\begin{pmatrix}
a_{1n}\\
a_{2n}\\
\vdots\\
a_{mn}
\end{pmatrix}
=
\begin{pmatrix}
b_1\\
b_2\\
\vdots\\
b_m
\end{pmatrix}
$$
$$
(*)\Leftrightarrow Ax=b
$$

В школе геометрической интерпретацией системы линейных уравнений (СЛУ) размера 3 на 3 было пересечение (необязательно) плоскостей. Для любых $m$ и $n$ геометрическая интерпретация есть пересечение  гиперплоскостей.

Система называется совместной, если она имеет хотя бы одно решение, и несовместной, если у неё нет ни одного решения. 
\begin{theorem}[Критерий Кронекера-Капелли]
Система совместна $\Leftrightarrow \Rg(A|b)=\Rg(A)$.
\end{theorem}
\section{Поиск решений}
Если матрица $A$ невырождена, то
\begin{align*}
Ax&=b \qquad A^{-1}\cdot |\\
x&=A^{-1}b
\end{align*}
\textbf{Метод Жордана-Гаусса}\\
$\exists T_1,\dots, T_M$ --- элементарные преобразования\\
$T_M\cdot \dots\cdot T_1 A=E \qquad |\cdot A^{-1}b$\\
$T_M\cdot \dots\cdot T_1 b=A^{-1}b=x$
\begin{prim}
	$$
	\begin{cases}
		x_1+x_2-x_3=7\\
		2x_1-3x_2+3x_3=4\\
		3x_1-x_2-2x_3=4
	\end{cases}
	$$
\end{prim}
Запишем расширенную матрицу СЛУ:\\
$
\left( \begin{array}{ccc|c}
	1 & 1 & -1 & 7\\
	2 & -3 & 3 & 4\\
	3 & -1 & -2 & 4\\
\end{array} \right)
\xrightarrow[(3)-3(1)]{(2)-2(1)}
\left( \begin{array}{ccc|c}
1 & 1 & -1 & 7\\
0 & -5 & 5 & -10\\
0 & -4 & 1 & -17\\
\end{array} \right)
\xrightarrow{(2)/(-5)}
\left( \begin{array}{ccc|c}
1 & 1 & -1 & 7\\
0 & 1 & -1 & 2\\
0 & -4 & 1 & -17\\
\end{array} \right)
\xrightarrow{(3)+4(2)}\\
\rightarrow\left( \begin{array}{ccc|c}
1 & 1 & -1 & 7\\
0 & 1 & -1 & 2\\
0 & 0 & -3 & -9\\
\end{array} \right)
\xrightarrow{(3)/(-3)}
\left( \begin{array}{ccc|c}
1 & 1 & -1 & 7\\
0 & 1 & -1 & 2\\
0 & 0 & 1 & 3\\
\end{array} \right)
\xrightarrow[(2)+(3)]{(1)+(3)}
\left( \begin{array}{ccc|c}
1 & 1 & 0 & 10\\
0 & 1 & 0 & 5\\
0 & 0 & 1 & 3\\
\end{array} \right)
\xrightarrow{(1)-(2)}
\left( \begin{array}{ccc|c}
1 & 0 & 0 & 5\\
0 & 1 & 0 & 5\\
0 & 0 & 1 & 3\\
\end{array} \right)
$

\vspace{2mm}
Ответ:$(x_1\ x_2\ x_3)^{\text{T}}=(5\ 5\ 3)^{\text{T}}$
\begin{prim}
	$$
	\left\{
	\begin{array}{rrrrrl}
	x_1&+3x_2&+3x_3&+2x_4&+6x_5&=0\\
	x_1&-x_2&-2x_3&&-3x_5&=0\\
	x_1&+11x_2&+7x_3&+6x_4&+18x_5&=0\\
	\end{array}
	\right.
	$$
\end{prim}
Данная система уравнений называется однородной. Она всегда совместна, т.к. имеет частное решение $x_i=0, i=\overline{1,5}$.\\
$
\left( \begin{array}{ccccc|c}
	1 & 3 & 3 & 2 & 6 & 0\\
	1 & -1 & -2 & 0 & -3 & 0\\
	1 & 11 & 7 & 6 & 18 & 0\\
\end{array} \right)
\xrightarrow[(3)-(1)]{(2)-(1)}
\left( \begin{array}{ccccc|c}
1 & 3 & 3 & 2 & 6 & 0\\
0 & -4 & -5 & -2 & -9 & 0\\
0 & 8 & 4 & 4 & 12 & 0\\
\end{array} \right)
\xrightarrow{(3)+2(2)}
\left( \begin{array}{ccccc|c}
1 & 3 & 3 & 2 & 6 & 0\\
0 & -4 & -5 & -2 & -9 & 0\\
0 & 0 & -6 & 0 & -6 & 0\\
\end{array} \right)
\rightarrow\\
\xrightarrow{(3)/(-6)}
\left( \begin{array}{ccccc|c}
1 & 3 & 3 & 2 & 6 & 0\\
0 & -4 & -5 & -2 & -9 & 0\\
0 & 0 & 1 & 0 & 1 & 0\\
\end{array} \right)
\xrightarrow[(1)-3(3)]{(2)+5(3)}
\left( \begin{array}{ccccc|c}
1 & 3 & 0 & 2 & 3 & 0\\
0 & -4 & 0 & -2 & -4 & 0\\
0 & 0 & 1 & 0 & 1 & 0\\
\end{array} \right)
\xrightarrow{(2)/(-4)}\\
\rightarrow
\left( \begin{array}{ccccc|c}
1 & 3 & 0 & 2 & 3 & 0\\
0 & 1 & 0 & 1/2 & 1 & 0\\
0 & 0 & 1 & 0 & 1 & 0\\
\end{array} \right)
\xrightarrow{(1)-3(2)}\\
\begin{matrix}
\phantom{ 1 8\,}
\overbrace{\ 
		\phantom{1 - 0 -}}^{\text{главные неизв.}}\\
\end{matrix}
$

\vspace{-15pt}
\noindent$\rightarrow
\left( \begin{array}{ccc|cc|c}\cline{4-5}
1 & 0 & 0 & 1/2 & 0 & 0\\ 
0 & 1 & 0 & 1/2 & 1 & 0\\
0 & 0 & 1 & 0 & 1 & 0\\\cline{4-5}
\end{array} \right)\\
$

\vspace{-10pt}
\noindent$
\phantom{0 + 1 8223}
\begin{matrix}
	\underbrace{\ 
		\phantom{ 0 1 0 6459}}_{\text{параметрические неизв.}}\\
\end{matrix}
$\\
Эта матрица эквивалентна системе
$$
\left\{
\begin{array}{rcl}
x_1 =& -\frac{1}{2}x_4&\\[1mm]
x_2 =& -\frac{1}{2}x_4&-x_5\\
x_3 =& \phantom{x_4}& -x_5
\end{array}
\right.
$$
Тогда\\
$
\begin{pmatrix}
	x_1\\
	x_2\\
	x_3\\
	x_4\\
	x_5\\
\end{pmatrix}
=
\begin{pmatrix}
-1/2 & 0\\
-1/2 & -1\\
0 & -1\\
1 & 0\\
0 & 1\\  
\end{pmatrix}
\begin{pmatrix}
c_1\\
c_2\\
\end{pmatrix}
=c_1
\begin{pmatrix}
-1/2 \\
-1/2 \\
0 \\
1 \\
0 \\  
\end{pmatrix}
+c_2
\begin{pmatrix}
0\\
-1\\
-1\\
0\\
1\\  
\end{pmatrix}, \qquad c_1,c_2 \in \mathbb{R}
$\\
Это и есть ответ к данной задаче.
$$
<\begin{pmatrix}
-1/2 \\
-1/2 \\
0 \\
1 \\
0 \\  
\end{pmatrix},
\begin{pmatrix}
0\\
-1\\
-1\\
0\\
1\\  
\end{pmatrix}>\text{--- линейная оболочка.}
$$
\vspace{2cm}
\begin{prim}
$$
	\left\{
	\begin{array}{rrrrrl}
		2x_1&-3x_2&-8x_3&+4x_4&-4x_5&=3\\
		 & 2x_2&&&+4x_5&=2\\
		-3x_1&+x_2&+12x_3&-6x_4&-x_5&=-8\\
		-x_1&-2x_2&+4x_3&-2x_4&-5x_5&=-5\\
	\end{array}
	\right.
$$
\end{prim}
$
\left( \begin{array}{ccccc|c}
	2 & -3 & 8 & 4 & -4 & 3\\
	0 & 2 & 0 & 0 & 4 & 2\\
	-3 & 1 & 12 & -6 & -1 & -8\\
	-1 & -2 & 4 & -2 & -5 & -5\\
\end{array} \right)
\xrightarrow[(4)\longleftrightarrow(1)]{(4)\times(-1);(2)/(2)}
\left( \begin{array}{ccccc|c}
1 & 2 & -4 & 2 & 5 & 5\\
0 & 1 & 0 & 0 & 2 & 1\\
-3 & 1 & 12 & -6 & -1 & -8\\
2 & -3 & 8 & 4 & -4 & 3\\
\end{array} \right)
\xrightarrow[(3)+3(1)]{(4)-2(1)}\\
\rightarrow
\left( \begin{array}{ccccc|c}
1 & 2 & -4 & 2 & 5 & 5\\
0 & 1 & 0 & 0 & 2 & 1\\
0 & 7 & 0 & 0 & 14 & 7\\
0 & -7 & 0 & 0 & -14 & -7\\
\end{array} \right)
\xrightarrow[(3)-7(2)]{(4)+7(2)}
\left( \begin{array}{ccccc|c}
1 & 2 & -4 & 2 & 5 & 5\\
0 & 1 & 0 & 0 & 2 & 1\\
0 & 0 & 0 & 0 & 0 & 0\\
0 & 0 & 0 & 0 & 0 & 0\\
\end{array} \right)
\xrightarrow{(1)-2(2)}
\left( \begin{array}{ccccc|c}
1 & 0 & -4 & 2 & 1 & 3\\
0 & 1 & 0 & 0 & 2 & 1\\
\end{array} \right)
$\\
Альтернатива Кронекера-Капелли:\\
Система несовместна $\Leftrightarrow$ она содержит строку $(0\ \cdots\ 0|1)$.\\
$
\begin{pmatrix}
	x_1\\
	x_2\\
	x_3\\
	x_4\\
	x_5\\
\end{pmatrix}
=
\begin{pmatrix}
3\\
1\\
0\\
0\\
0\\
\end{pmatrix}
+
\begin{pmatrix}
4 & -2 & -1\\
0 & 0 & -2\\
1 & 0 & 0\\
0 & 1 & 0\\
0 & 0 & 1\\
\end{pmatrix}
\begin{pmatrix}
c_1\\
c_2\\
c_3\\
\end{pmatrix}
=
\begin{pmatrix}
3\\
1\\
0\\
0\\
0\\
\end{pmatrix}
+c_1
\begin{pmatrix}
4\\
0 \\
1 \\
0 \\
0 \\
\end{pmatrix}
+c_2
\begin{pmatrix}
-2\\
0\\
0\\
1\\
0\\
\end{pmatrix}
+c_3
\begin{pmatrix}
-1\\
-2\\
0\\
0\\
1\\
\end{pmatrix}
, \qquad c_1, c_2, c_3 \in \mathbb{R}.
$\\
Запишем фундаментальную систему решений (решение однородной системы):
$$
\text{Ф}=
\begin{pmatrix}
4 & -2 & -1\\
0 & 0 & -2\\
1 & 0 & 0\\
0 & 1 & 0\\
0 & 0 & 1\\
\end{pmatrix}
$$

Матрица ФСР выдерживает элементарные преобразования столбцов.

Будем говорить об однородной СЛУ. Столбцы ФСР --- базис в пространстве решений однородной системы.

Пусть $\text{Ф}'_{m\times n},\ \text{Ф}_{m\times n}$ --- ФСР системы $Ax=0$. $$\exists C_{n\times n}: C \text{--- неврожденная } \& \text{ Ф}'_{m\times n}=\text{Ф}_{m\times n} C.$$

Число столбцов ФСР = числу свободных неизвестных = число всех неизвестных ($n$) - число главных неизвестных ($\Rg A$). 
\begin{prim}
	$$
	\left\{
	\begin{array}{rrrrrl}
		3x_1&+2x_2&+x_3&&=7\\
		-4x_1&+5x_2&&+x_4&=3\\
	\end{array} \right.
	$$
\end{prim}
$
\left( \begin{array}{cccc|c}
3 & 2& 1 & 0& 7\\
-4 & 5& 0 & 1& 3\\
\end{array}\right)
$\\
Очевидно, что ничего преобразовывать не надо, единичная матрица уже есть.\\
$$
\begin{pmatrix}
x_1\\
x_2\\
x_3\\
x_4\\
\end{pmatrix}
=
\begin{pmatrix}
0\\
0\\
7\\
3\\
\end{pmatrix}
+c_1
\begin{pmatrix}
1\\
0\\
-3\\
4\\
\end{pmatrix}
+c_2
\begin{pmatrix}
0\\
1\\
-2\\
-5\\
\end{pmatrix}
, \qquad c_1, c_2, c_3 \in \mathbb{R}.
$$

\begin{prim}
	Дана ФСР
	$$
	\text{Ф}=
	\begin{pmatrix}
	1&0\\
	1&-1\\
	0&1\\
	-4&-1\\
	\end{pmatrix}
	$$
	Найти $Ax=0$.
\end{prim}

Припишем справа столбец неизвестных. Т.к. он линейно выражается через строки матрицы $A$, то это не изменит ранг нашей матрицы.\\
$
\left( \begin{array}{cc|c}
1&0&x_1\\
1&-1&x_2\\
0&1&x_3\\
-4&-1&x_4\\
\end{array}\right)
\xrightarrow{(2)\longleftrightarrow(3)}
\left( \begin{array}{cc|c}
1&0&x_1\\
0&1&x_3\\
1&-1&x_2\\
-4&-1&x_4\\
\end{array}\right)
\xrightarrow[(4)+4(1)]{(3)-(1)}
\left( \begin{array}{cc|c}
1&0&x_1\\
0&1&x_3\\
0&-1&x_2-x_1\\
0&-1&x_4+4x_1\\
\end{array}\right)
\xrightarrow[(3)+(2)]{(4)+(2)}\\
\rightarrow
\left( \begin{array}{cc|c}
1&0&x_1\\
0&1&x_3\\
0&0&x_2-x_1+x_3\\
0&0&x_4+4x_1+x_3\\
\end{array}\right)
$\\
Ответ: $\left\{ \begin{array}{rrrrl}
	-x_1&+x_2&+x_3&&=0\\
	4x_1&&+x_3&+x_4&=0\\
\end{array}
\right.$
\begin{prim}
	При каких $\alpha$ и $\beta$ система совместна? Решить ее.
	$$\left\{ \begin{array}{rrrrl}
	2x_1&-4x_2&+3x_3&-2x_4&=3\\
	&&x_3&-2x_4&=\alpha\\
	3x_1&-6x_2&+2x_3&+2x_4&=2\\
	-3x_1&+6x_2&-x_3&-4x_4&=\beta\\
	\end{array}
	\right.$$
\end{prim}

$
\left( \begin{array}{cccc|c}
2&-4&3&-2&3\\
0&0&1&-2&\alpha\\
3&-6&2&2&2\\
-3&6&-1&-4&\beta\\
\end{array}\right)
\xrightarrow[(3)\cdot 2; (4)\cdot 2]{(1)\cdot 3}
\left( \begin{array}{cccc|c}
6&-12&9&-6&9\\
0&0&1&-2&\alpha\\
6&-12&4&4&4\\
-6&12&-2&-8&2\beta\\
\end{array}\right)
\xrightarrow[(4)+(1);(1)/3]{(3)-(1)}\\
\rightarrow
\left( \begin{array}{cccc|c}
2&-4&3&-2&3\\
0&0&1&-2&\alpha\\
0&0&-5&10&-5\\
0&0&7&-14&2\beta+9\\
\end{array}\right)
\qquad \Rg A = \Rg(A|b) = 2 \Rightarrow \alpha = 1, \qquad 2\beta+9=7
$\\
То есть $\alpha = 1, \beta=-1$.\\
$
\left( \begin{array}{cccc|c}
2&-4&3&-2&3\\
0&0&1&-2&1\\
\end{array}\right)
\xrightarrow{(1)-3(2)}
\left( \begin{array}{cccc|c}
2&-4&0&4&0\\
0&0&1&-2&1\\
\end{array}\right)
\xrightarrow{(1)/2}
\left( \begin{array}{cccc|c}
1&-2&0&2&0\\
0&0&1&-2&1\\
\end{array}\right)
\rightarrow\\
\rightarrow
\left( \begin{array}{cccc|c}
1&0&-2&2&0\\
0&1&0&-2&1\\
\end{array}\right)\\
\begin{pmatrix}
x_1\\
x_3\\
x_2\\
x_4\\
\end{pmatrix}
=
\begin{pmatrix}
0\\
1\\
0\\
0\\
\end{pmatrix}
+
\begin{pmatrix}
2&-2\\
0&2\\
1&0\\
0&1\\
\end{pmatrix}
\begin{pmatrix}
c_1\\
c_2\\
\end{pmatrix}\\
$
$$
\begin{pmatrix}
x_1\\
x_2\\
x_3\\
x_4\\
\end{pmatrix}
=
\begin{pmatrix}
0\\
0\\
1\\
0\\
\end{pmatrix}
+c_1
\begin{pmatrix}
2\\
1\\
0\\
0\\
\end{pmatrix}
+c_2
\begin{pmatrix}
-2\\
0\\
2\\
1\\
\end{pmatrix}
$$
\subsection{Примечание}
Обратимся к примеру 3. Мы получили такую расширенную матрицу:
$$
\left( \begin{array}{ccccc|c}
1 & 0 & -4 & 2 & 1 & 3\\
0 & 1 & 0 & 0 & 2 & 1\\
\end{array} \right)
$$
Она эквивалентна системе:
$$
\left\{
\begin{array}{rrrrrrl}
x_1&&-4x_3&+2x_4&+x_5&=3\\
&x_2&&&+2x_5&=1,\\
\end{array}
\right.
$$
откуда
$$
\left\{
\begin{array}{rl}
x_1&=3+4x_3-2x_4-x_5\\
x_2&=1-2x_5.\\
\end{array}
\right.
$$
Тогда можно записать:
$$
\begin{pmatrix}
x_1\\
x_2\\
x_3\\
x_4\\
x_5\\
\end{pmatrix}
=
\begin{pmatrix}
3+4x_3-2x_4-x_5\\
1-2x_5\\
x_3\\
x_4\\
x_5\\
\end{pmatrix}
=
\begin{pmatrix}
3\\
1\\
0\\
0\\
0\\
\end{pmatrix}
+x_3
\begin{pmatrix}
4\\
0\\
1\\
0\\
0\\
\end{pmatrix}
+x_4
\begin{pmatrix}
-2\\
0\\
0\\
1\\
0\\
\end{pmatrix}
+x_5
\begin{pmatrix}
-1\\
-2\\
0\\
0\\
1\\
\end{pmatrix}
,\qquad x_3, x_4, x_5 \in \mathbb{R}.
$$
Так мы пришли к ответу, не прибегая к умножению матриц.