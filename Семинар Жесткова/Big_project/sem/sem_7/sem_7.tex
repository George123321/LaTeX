\section{Инвариантные подпространства}
\underline{Будем работать только с преобразованиями.} %22:15
\subsection{Определение}
\begin{definition}
Подпространство $L' \subset L$ называется \textsf{инвариантным} \textbf{относительно преобразования $\bm\phi$}, если
$$
\forall \textbf{x} \in L' \mapsto \phi(\textbf{x}) \in L'\text{ или }\phi(L')\subset L'.
$$
\end{definition}
Например:
\begin{itemize}
	\item \textbf{o}, L --- вырожденные случаи.
	\item Поворот $\mathbb{R}^3$ вокруг \textbf{e$_3$} на $\pi/2$ (рис. \ref{prim1}). Инвариантные подпространства: $\textbf{o}, \mathbb{R}^3, \langle \textbf{e$_1$}, \textbf{e$_2$} \rangle, \langle \textbf{e$_3$}\rangle$.
	\item Ядро преобразования $\phi \ (\ker \phi)$ всегда инвариантно относительно этого преобразования $\phi$.
	\item $\im \phi$
	\begin{theorem}
	Если какое-то подпространство содержит в себе образ, то $L'$ инвариантно относительно $\phi$.
	\end{theorem}
	\begin{proof}
		$$\forall \textbf{x} \in L' \mapsto \phi(\textbf{x}) \subset \im\phi\subset L'$$
	\end{proof}
\end{itemize}
\subsection{Свойства инвариантных подпространств}
\begin{predlog}
	Сумма инвариантных подпространств инвариантна.
\end{predlog}
\begin{proof}
	$$
	\left.
	\begin{array}{r}
		\textbf{x} \in L_1, \phi(\textbf{x}) \in L_1\\
		\textbf{y} \in L_2, \phi(\textbf{y}) \in L_2\\
	\end{array}
	\right\} \Rightarrow \phi(\textbf{x}) + \phi(\textbf{y}) \in L_1 + L_2
	$$
\end{proof}
\begin{predlog}
	Пересечение инвариантных подпространств инвариантно.
\end{predlog}
\begin{proof}
	$$
	\left.
	\begin{array}{r}
	\textbf{x} \in L_1, \phi(\textbf{x}) \in L_1\\
	\textbf{x} \in L_2, \phi(\textbf{x}) \in L_2\\
	\end{array}
	\right\} \Rightarrow \phi(\textbf{x}) \in L_1 \cap L_2
	$$
\end{proof}
\section{Матрица преобразования}
\subsection{Вид матрицы преобразования}
Рассмотрим линейное пространство $L$, $\dim L=n$. Пусть $L' \subset L$, $\dim L' = k$ --- инвариантное подпространство относительно $\phi$, базис в $L': \{\textbf{e$_1$}, \dots, \textbf{e$_k$}\}$, базис в $L: \{\textbf{e$_1$}, \dots, \textbf{e$_k$}, \textbf{e$_{k+1}$}, \dots, \textbf{e$_n$}\}$. %22:44 23:10

\begin{figure}
	\centering
	\def\svgwidth{5cm} % если надо изменить размер
	\input{sem/sem_7/podpr.pdf_tex}
	\caption{Подпространство в L}
	\label{podpr}
\end{figure}

Напомним, что матрица преобразования $A$ строится из образов базисных векторов:
$$A=
\left(\hspace{1mm}
\begin{array}{|c|c|c|}
\cline{1-1} \cline{3-3}
\multirow{4}{*}{$\phi(e_1)$} & \multirow{4}{*}{$\cdots$} & \multirow{4}{*}{$\phi(e_n)$} \\
&                           &                              \\
&                           &                              \\
&                           &                              \\ \cline{1-1} \cline{3-3} 
\end{array}\hspace{1mm}\right)
$$
Т.о. матрица преобразования в выбранном базисе имеет следующий вид:
$$
A =\left(
\begin{tabular}{c|c}
$A_1$ & $A_2$\\
\hline
$O$ & $A_4$\\
\end{tabular}
\right)^{\square}\text{ --- \textsf{клетчочно--треугольный вид}.}\footnote{Квадрат над матрицей значит, что матрица блочная.}
$$

Пусть теперь $L=L_1\oplus L_2 \oplus \dots \oplus L_s, \forall i\ L_i$ --- инвариантное подпространство относительно $\phi$.

\begin{figure}[h!]
	\centering
	\def\svgwidth{10cm} % если надо изменить размер
	\input{sem/sem_7/summa.pdf_tex}
	\caption{Прямая сумма подпространств}
	\label{summa}
\end{figure}

Тогда матрица преобразования имеет вид:
$$
A =\left(
\begin{array}{cccc}\cline{1-1}
\multicolumn{1}{|c|}{A_1} & O & \dots & O\\\cline{1-1} \cline{2-2}
O & \multicolumn{1}{|c|}{A_2} &\dots& O\\\cline{2-2}
\vdots&\vdots&\ddots&\vdots\\\cline{4-4}
O & O &\dots & \multicolumn{1}{|c|}{A_s}\\\cline{4-4}
\end{array}
\right)^{\square}\text{ --- \textsf{клеточно--диагноальный вид}.}
$$
Ширина и высота каждой клетки равны размерности инвариантного подпространства.
\begin{prim} %\begin{prim} !!!
	Найти инвариантные подпространства в $\mathbb{R}^3$ относительно $\varphi$. %подпрострашнства, R^3!!!!
\end{prim}\\

\begin{wrapfigure}{r}{0.27\linewidth}
	\def\svgwidth{3cm} % если надо изменить размер
	\input{sem/sem_7/prim2.pdf_tex}
	\caption{К примеру 1}
	\label{prim1}
	\vspace{-1cm}
\end{wrapfigure}

$A = \left( \begin{array}{rr|r} % ВЫРАВНИВАНИЕ
0&-1&0\\
1&0&0\\ 
\hline
0&0&1
\end{array} \right)$

Из вида $A$ видно, что существует два не пересекающихся инвариантных подпространства.\\
$\textbf{o}, \mathbb{R}^3, \langle \textbf{e$_1$}, \textbf{e$_2$} \rangle, \langle \textbf{e$_3$} \rangle.$\\ % Векторы жирным!!!!!  

\subsection{Геометрический смысл матрицы преобразования}
Поговорим о геометрии. Научимся определять по внешнему виду матрицы ее геометрический смысл.

$\left( \begin{array}{rrr}
1 & 0 & 0  \\
0 & 1& 0\\
0&0&-1
\end{array}\right)$~---~отражение относительно $\langle \textbf{e$_1$}, \textbf{e$_2$} \rangle.$ %Точки в конце предложений надо ставить!

$\left( \begin{array}{rrr}
1 & 0 & 0  \\
0 & 1& 0\\
0&0&0
\end{array}\right)$~---~проекция.
%Меньше \vspace{}

$\left( \begin{array}{rrr}
1 & 0 & 0  \\
0 & 3& 0\\
0&0&1
\end{array}\right)$~---~растяжение вдоль $\textbf{e$_2$}$ в 3 раза.\\

Нам интересны матрицы вида:

$A=\left( \begin{array}{ccc} %Здесь выравниваение пришлось оставить из-за того, что выравниваение по правому краю происходит по индексам, что выглядит ужасно
\lambda_1 & 0 & 0  \\
0 & \lambda_2& 0\\
0&0&\lambda_3
\end{array}\right)$~---~ обобщённое растяжение $\Leftrightarrow \varphi(\textbf{e$_i$}) = \lambda_i\textbf{e$_i$}$.

\section{Собственный вектор}
\subsection{Определение}
Рассмотрим преобразование  $\varphi$ с матрицей $A$, тогда ненулевой вектор $\textbf{x}$ называется \textsf{собственным вектором}, если  $\varphi(\textbf{x}) = \lambda \textbf{x}$; $\lambda$~---~собственное значение. %Точки в конце предложений!!! + лучше выделять не почеркиванием, а рубленым шрифтом + векторы жирным!!!

Множество собственных векторов, отвечающих одному и тому же собственному значению, образует \textsf{собственное пространство}.
\subsection{Свойства}
\begin{predlog} % Оформление под теорему!
	Собственный вектор (и только он) порождает одномерное инвариантное подпространство. % "Собственный вектор (и только он) порождает одномерное инвариантно пространство" - я конечно все понимаю, но "инвариантно пространство" ни в какие ворота, так еще и породить пространство он ну никак не может.
\end{predlog}
\begin{proof}
	Рассмотрим инвариантное подпространство  $\langle \textbf{x} \rangle \Rightarrow \varphi(\alpha \textbf{x}) = \alpha \varphi(\textbf{x}) = \alpha\lambda \textbf{x} \in~\langle \textbf{x}\rangle$.
\end{proof}
\section{Алгоритм поиска собственных значений и собственных векторов}
\vspace{-0.5cm}
$$\varphi(\textbf{x}) = \lambda \textbf{x}$$
$$\varphi(\textbf{x}) - \lambda \textbf{x} = \textbf{o}$$
Рассмотрим тождественное преобразование $Id$, матрица его преобразования $E$. % Без сокращений!!!
$$(\varphi -\lambda Id)(\textbf{x})=\textbf{o}$$
Перейдём к матричному виду:
\begin{equation}
(\varphi -\lambda E)\textbf{x}=\textbf{o} \tag{$*$}
\end{equation}
Итак, мы получили СЛУ размеров $n\times n$. Она имеет либо одно решение (нулевое), но оно нам не интересно, т.~к. $\textbf{x}\neq\textbf{o}$, либо бесконечно много решений $\Rightarrow A$ должна быть вырожденной $\Rightarrow \det(A-\lambda E) = 0 \rightarrow \lambda_i\text{ --- cобственное значение} \rightarrow (*) \rightarrow \textbf{x}\text{ --- cобственный вектор}.$ %ТОЧКА!, тире ---!!!!
\begin{prim}
	Найти собственные значения и собственные векторы.\\
	$$A = \left( \begin{array}{rrr}
	2 & 2 & 1  \\
	-2 & -3& 2\\
	3&6&0
	\end{array}\right)
	$$
\end{prim}\\

Найдём $\lambda$ из условия $\det(A-\lambda E) = 0$:

$\begin{array}{|ccc|}
2-\lambda & 2 & 1  \\
-2 & -3 -\lambda& 2\\
3&6&-\lambda
\end{array} = 0 = -\lambda^3- \lambda^2+17\lambda -15 \Rightarrow \lambda_1 =1, \lambda_2 =3, \lambda_3 = -5$\\
Далее подставим числа $\lambda$ в $(*)$:

\begin{enumerate}
	\item{
		$\lambda_1=1 :$
		$\begin{pmatrix*}[r]
		1 & 2 & 1 & \vrule & 0\\
		-2 & -4 & 2 & \vrule & 0\\
		3 & 6 & -1 & \vrule & 0\\
		\end{pmatrix*} , $\\
		$
		L_1=
		\left(
		\begin{smallmatrix*}[r]
		x_1\\ x_2\\ x_3\\ 
		\end{smallmatrix*}
		\right) 
		=
		\langle
		\left(
		\begin{smallmatrix*}[r]
		2\\ -1\\ 0\\ 
		\end{smallmatrix*}
		\right) 
		\rangle
		$
		$\leftarrow $
		Собственный вектор, перейдёт сам в себя т.к. $\lambda=1$.
	}
	\item{
		$\lambda_2=3 :
		L_2=
		\left(
		\begin{smallmatrix*}[r]
		x_1\\ x_2\\ x_3\\ 
		\end{smallmatrix*}
		\right) 
		=
		\langle
		\left(
		\begin{smallmatrix*}[r]
		1\\ 0\\ 1\\ 
		\end{smallmatrix*}
		\right) 
		\rangle
		$
	}
	\item{
		$\lambda_3=-5 :
		L_3=
		\left(
		\begin{smallmatrix*}[r]
		x_1\\ x_2\\ x_3\\ 
		\end{smallmatrix*}
		\right) 
		=
		\langle
		\left(
		\begin{smallmatrix*}[r]
		1\\ -8\\ 9\\ 
		\end{smallmatrix*}
		\right) 
		\rangle
		$
	}
\end{enumerate}

\begin{lemma} % окружение лемма!
	Собственные векторы, соответствующие различным собственным значениям, попарно линейно независимы.
\end{lemma}


Соберём базис $\textbf{f}$ %f в мат режиме
$  %для красоты пришлось сделать ужас: в матрице матрица, у которой в подписи матрица.
\begin{matrix}
\overset{\left\{
	\begin{pmatrix*}[r]
	2\\ -1\\ 0\\ 
	\end{pmatrix*},
	\begin{pmatrix*}[r]
	1\\ 0\\ 1\\ 
	\end{pmatrix*},
	\begin{pmatrix*}[r]
	1\\ -8\\ 9\\ 
	\end{pmatrix*}
	\right\}}{
	\begin{matrix*}[r]
	\textbf{f$_1$}\phantom{3}&\textbf{f$_2$}&\phantom{2}\textbf{f$_3$}
	\end{matrix*}}
\end{matrix}
$\\
$
\left.
\begin{aligned} %раз уж aligned, сделай выравнивание
\varphi (\textbf{f$_1$})&=\textbf{f$_1$}\\ %ВЕКТОРЫ
\varphi (\textbf{f$_2$})&=3\textbf{f$_2$} \\
\varphi (\textbf{f$_3$})&=5\textbf{f$_3$}\\
\end{aligned}
\right.
\to A'=
\begin{pmatrix*}[r]
1 & 0 & 0\\
0 & 3 & 0\\
0 & 0 & -5\\
\end{pmatrix*}
$
в базисе $\textbf{f}$.\\ %точки в конце предложений!

\section{Диагонализируемость матрицы}
\begin{prim}
	Диагонализировать матрицу\\
	$$
	A=
	\begin{pmatrix*}[r]
	3 & 1 & -2\\
	2 & 2 & -2\\
	2 & 1 & -1\\
	\end{pmatrix*}
	$$
\end{prim}\\

$
\det(A-\lambda E)=0$:\\ %заебался уже говорить о мат функциях, об их оформлениях. Почему нельзя делать правильно - я не понимаю.
$
\begin{vmatrix*}[c] %лучше сделать центрирование
3-\lambda & 1 & -2\\
2 & 2-\lambda & -2\\
2 & 1 & -1-\lambda\\
\end{vmatrix*}
=0
\leftrightarrow
\underset{ \text{Не забывайте про свойства детерминанта}}{(\lambda-1)^2(\lambda-2)=0}\\
$
$\lambda =1$ --- корень алгебраической кратности 2.\\ % тире тройное!!!!!!!!!!!!+что такое алгоритмическая кратность?
$\lambda =2$ --- корень алгебраической кратности 1 (простой корень).\\
\begin{enumerate}
	\item{
		$\lambda =1   $ \\ 
		$
		\begin{pmatrix*}[r]
		2 & 1 & -2 & \vrule & 0\\
		2 & 1 & -2 & \vrule & 0\\
		2 & 1 & -2 & \vrule & 0\\
		\end{pmatrix*} ,
		$
		$
		L_1=
		\left(
		\begin{smallmatrix*}[r]
		x_1\\ x_2\\ x_3\\ 
		\end{smallmatrix*}
		\right) 
		=
		\langle
		\underbrace{
			\left(
			\begin{smallmatrix*}[r] % в матрицах лучше не пользоваться frac
			-1/2\\ -1\\ 0\\ 
			\end{smallmatrix*}
			\right) ,
			\left(
			\begin{smallmatrix*}[r]
			1\\ 0\\ 1\\ 
			\end{smallmatrix*}
			\right) 
		}_{\substack{\text{формирует}\\\text{плоскость}}}
		\rangle
		$\\
		$ \dim L_1=2$ --- геометрическая кратность (размерность собственного подпространства). % мат функции
	}
	\item $\lambda = 2$ $$\Longrightarrow
	\begin{pmatrix}
	1 & 1 & -2 & \vrule & 0\\
	2 & 0 & -2 & \vrule & 0\\
	2 & 1 & -3 & \vrule & 0\\
	\end{pmatrix}
	\Rightarrow
	~L_1 = 
	\langle
	\left(
	\begin{smallmatrix*}[c]
	x_1\\ x_2\\ x_3\\ 
	\end{smallmatrix*}
	\right) 
	\rangle=
	\langle
	\left(
	\begin{smallmatrix*}[c]
	1\\ 1\\ 1\\ 
	\end{smallmatrix*}
	\right) 
	\rangle $$
\end{enumerate}
Выберем базис: $\left\{
\left(
\begin{smallmatrix*}[c]
-1/2\\ 1\\ 0\\
\end{smallmatrix*}
\right) 
,
\left(
\begin{smallmatrix*}[c]
1\\ 0\\ 1\\
\end{smallmatrix*}
\right) 
,
\left(
\begin{smallmatrix*}[c]
1\\ 1\\ 1\\
\end{smallmatrix*}
\right) 
\right\}$\\

Поэтому $\varphi(\textbf{f$_1$})=\textbf{f$_1$},~\varphi(\textbf{f$_2$})=\textbf{f$_2$},~\varphi(\textbf{\textbf{f$_3$}})=2\textbf{f$_3$}$\\ % ВЕКТОРЫ!

\textbf{Ответ:} $A'=\begin{pmatrix}
1 & 0 & 0\\
0 & 1 & 0\\
0 & 0 & 2\\
\end{pmatrix}.$

\bigskip

$\bullet$ Геометрическая кратность $\le$ алгебраическая кратность\\

$\bullet$ Если геометрическая кратность строго меньше ($<$) алгебраической кратности хотя бы для одного $\lambda$, то преобразование \textsf{недиагонализируемо}.

\begin{prim}
	Диагонализировать матрицу:~~$A=\begin{pmatrix}
	2 & 1 & 0\\
	0 & 2 & 1\\
	0 & 0 & 2\\
	\end{pmatrix}$\\
\end{prim}

$$
\det~(A - \lambda E) = 0 \Rightarrow
\begin{vmatrix*}[c]
2-\lambda & 1 & 0\\
0 & 2-\lambda  & 1\\
0 & 0  & 2-\lambda\\
\end{vmatrix*}
= (2-\lambda)^3= 0 $$

Получаем, что $\lambda = 2$ алгебраической кратности 3.

$$\Longrightarrow
\begin{pmatrix}
0 & 1 & 0 & \vrule & 0\\
0 & 0 & 1 & \vrule & 0\\
0 & 0 & 0 & \vrule & 0\\
\end{pmatrix};
~L_1 = 
\langle
\left(
\begin{smallmatrix*}[c]
x_1\\ x_2\\ x_3\\ 
\end{smallmatrix*}
\right) 
\rangle=
\langle
\left(
\begin{smallmatrix*}[c]
1\\ 0\\ 0\\ 
\end{smallmatrix*}
\right) 
\rangle $$

Получили, что геометрическая кратность (равна 1) меньше алгебраической кратности (равна 3). Тогда матрица недиагонализируема (не хватило собственных векторов).    
\newpage
\begin{prim}
	Диагонализировать матрицу:
	
	$$A =
	\begin{pmatrix*}[r]
	0 & -1 & 0\\
	1 & 0  & 0\\
	0 & 0  & 1\\
	\end{pmatrix*}\\
	$$	
\end{prim}\\

$
\det~(A - \lambda E) = 0 \Rightarrow % мат функции
\begin{vmatrix*}[c] % лучше центрирование
-\lambda & -1 & 0\\
1 & -\lambda  & 0\\
0 & 0  & 1-\lambda\\
\end{vmatrix*}
= (\lambda - 1)(\lambda^{2} + 1) = 0 $
\\
$\lambda = 1, ~ \underbrace {\lambda = \pm ~i}_
{\substack{
		\text{отвечают за}\\
		\text{инвариантную}\\
		\text{плоскость} }} \\
\\
\\
\begin{pmatrix*}[r]
-1 & -1 & 0 & \vrule & 0\\
1 & -1 & 0 & \vrule & 0\\
0 & 0 & 0 & \vrule & 0\\
\end{pmatrix*};
~L_1 = 
\langle
\left(
\begin{smallmatrix*}[r]
0\\ 0\\ 1\\ 
\end{smallmatrix*}
\right) 
\rangle $
\\
\\
\\
\textbf {Условие диагонализируемости матрицы:} 
\begin{enumerate}
	\item {\itshape В частности}: $A_{n \times n}$ диагонализируема, если $A$ имеет $n$ различных вещественных собственных значений. % перед двоеточием пробелов не ставят
	\item {\itshape В общем случае}: $A$ диагонализируема $\Leftrightarrow$ $L$ раскладывается в прямую сумму собственных подпространств. % тройное тире!, да и вообще оно тут и не нужно
	
\end{enumerate}


























