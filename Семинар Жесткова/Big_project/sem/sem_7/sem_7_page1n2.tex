\section{Инвариантные подпространства}
\underline{Будем работать только с преобразованиями.} %22:15
\subsection{Определение}
\begin{definition}
Подпространство $L' \subset L$ называется \textsf{инвариантным} \textbf{относительно преобразования $\bm\phi$}, если
$$
\forall \textbf{x} \in L' \mapsto \phi(\textbf{x}) \in L'\text{ или }\phi(L')\subset L'.
$$
\end{definition}
Например:
\begin{itemize}
	\item \textbf{o}, L --- вырожденные случаи.
	\item Поворот $\mathbb{R}^3$ вокруг \textbf{e$_3$} на $\pi/2$ (рис. \ref{prim1}). Инвариантные подпространства: $\textbf{o}, \mathbb{R}^3, \langle \textbf{e$_1$}, \textbf{e$_2$} \rangle, \langle \textbf{e$_3$}\rangle$.
	\item Ядро преобразования $\phi \ (\ker \phi)$ всегда инвариантно относительно этого преобразования $\phi$.
	\item $\im \phi$
	\begin{theorem}
	Если какое-то подпространство содержит в себе образ, то $L'$ инвариантно относительно $\phi$.
	\end{theorem}
	\begin{proof}
		$$\forall \textbf{x} \in L' \mapsto \phi(\textbf{x}) \subset \im\phi\subset L'$$
	\end{proof}
\end{itemize}
\subsection{Свойства инвариантных подпространств}
\begin{predlog}
	Сумма инвариантных подпространств инвариантна.
\end{predlog}
\begin{proof}
	$$
	\left.
	\begin{array}{r}
		\textbf{x} \in L_1, \phi(\textbf{x}) \in L_1\\
		\textbf{y} \in L_2, \phi(\textbf{y}) \in L_2\\
	\end{array}
	\right\} \Rightarrow \phi(\textbf{x}) + \phi(\textbf{y}) \in L_1 + L_2
	$$
\end{proof}
\begin{predlog}
	Пересечение инвариантных подпространств инвариантно.
\end{predlog}
\begin{proof}
	$$
	\left.
	\begin{array}{r}
	\textbf{x} \in L_1, \phi(\textbf{x}) \in L_1\\
	\textbf{x} \in L_2, \phi(\textbf{x}) \in L_2\\
	\end{array}
	\right\} \Rightarrow \phi(\textbf{x}) \in L_1 \cap L_2
	$$
\end{proof}
\section{Матрица преобразования}
\subsection{Вид матрицы преобразования}
Рассмотрим линейное пространство $L$, $\dim L=n$. Пусть $L' \subset L$, $\dim L' = k$ --- инвариантное подпространство относительно $\phi$, базис в $L': \{\textbf{e$_1$}, \dots, \textbf{e$_k$}\}$, базис в $L: \{\textbf{e$_1$}, \dots, \textbf{e$_k$}, \textbf{e$_{k+1}$}, \dots, \textbf{e$_n$}\}$. %22:44 23:10

\begin{figure}
	\centering
	\def\svgwidth{5cm} % если надо изменить размер
	\input{podpr.pdf_tex}
	\caption{Подпространство в L}
	\label{podpr}
\end{figure}

Напомним, что матрица преобразования $A$ строится из образов базисных векторов:
$$A=
\left(\hspace{1mm}
\begin{array}{|c|c|c|}
\cline{1-1} \cline{3-3}
\multirow{4}{*}{$\phi(e_1)$} & \multirow{4}{*}{$\cdots$} & \multirow{4}{*}{$\phi(e_n)$} \\
&                           &                              \\
&                           &                              \\
&                           &                              \\ \cline{1-1} \cline{3-3} 
\end{array}\hspace{1mm}\right)
$$
Т.о. матрица преобразования в выбранном базисе имеет следующий вид:
$$
A =\left(
\begin{tabular}{c|c}
$A_1$ & $A_2$\\
\hline
$O$ & $A_4$\\
\end{tabular}
\right)^{\square}\text{ --- \textsf{клетчочно--треугольный вид}.}\footnote{Квадрат над матрицей значит, что матрица блочная.}
$$

Пусть теперь $L=L_1\oplus L_2 \oplus \dots \oplus L_s, \forall i\ L_i$ --- инвариантное подпространство относительно $\phi$.

\begin{figure}[h!]
	\centering
	\def\svgwidth{10cm} % если надо изменить размер
	\input{summa.pdf_tex}
	\caption{Прямая сумма подпространств}
	\label{summa}
\end{figure}

Тогда матрица преобразования имеет вид:
$$
A =\left(
\begin{array}{cccc}\cline{1-1}
\multicolumn{1}{|c|}{A_1} & O & \dots & O\\\cline{1-1} \cline{2-2}
O & \multicolumn{1}{|c|}{A_2} &\dots& O\\\cline{2-2}
\vdots&\vdots&\ddots&\vdots\\\cline{4-4}
O & O &\dots & \multicolumn{1}{|c|}{A_s}\\\cline{4-4}
\end{array}
\right)^{\square}\text{ --- \textsf{клеточно--диагноальный вид}.}
$$
Ширина и высота каждой клетки равны размерности инвариантного подпространства.


























