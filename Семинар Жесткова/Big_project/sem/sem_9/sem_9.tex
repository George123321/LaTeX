\chapter{Билинейные и квадратичные функции.}
\section{Билинейные функции}
Билинейной функцией\footnote{Здесь и далее будем использовать слово <<функция>>, при этом отметим, что слово <<\textbf{форма}>> в рассматриваемом смысле означает то же самое} на линейном пространстве $\mathcal{L}$ (над полем действительных чисел)  называется функция $\mathbf{b}$ от двух векторов из $\mathcal{L}$, линейная по каждому из своих аргументов, т.е. для любых $x$, $y$, $z$ из  $\mathcal{L}$ и для любого числа $\alpha$ верно:

\begin{itemize}
\item $\mathbf{b}(x+y,z)=\mathbf{b}(x,z)+\mathbf{b}(y,z)$,
\item $\mathbf{b}(\alpha x,y)=\alpha \mathbf{b}(x,y)$,
\item $\mathbf{b}(x,y+z)=\mathbf{b}(x,y)+\mathbf{b}(x,z)$,
\item $\mathbf{b}( x,\alpha y)=\alpha \mathbf{b}(x,y)$.
\end{itemize}

Например, известное из курса аналитической геометрии скалярное произведение векторов $(\mathbf{x},\mathbf{y})$ является билинейной функцией, в силу известных свойств. Другим примером билинейной функции может являться смешанное произведение векторов $(\mathbf{a},\mathbf{x},\mathbf{y})$, где $\mathbf{a}$ -- некоторый фиксированный вектор.\\

Зафиксируем в пространстве $\mathcal{L}$ базис $\mathbf{e}=\Vert e_1 \ldots e_n\Vert$. Пусть в этом базисе вектор $x$ имеет координатный столбец $\bm{\xi}=(\xi_1\ldots\xi_n)^{\text{T}}$, а вектор $y$ имеет координатный столбец $\bm{\eta}=(\eta_1\ldots\eta_n)^{\text{T}}$. Иначе говоря:$$x=\xi_1e_1+\xi_2e_2+\ldots+\xi_ne_n=\mathbf{e}\bm{\xi}$$$$
y=\eta_1e_1+\eta_2e_2+\ldots+\eta_ne_n=\mathbf{e}\bm{\eta}$$

Рассмотрим теперь билинейную функцию от векторов $x$ и $y$
$$\mathbf{b}(x,y)=\mathbf{b}(\xi_1e_1+\xi_2e_2+\ldots+\xi_ne_n,\hspace{5pt} \eta_1e_1+\eta_2e_2+\ldots+\eta_ne_n).$$

Воспользовавшись теперь указанными в определении свойствами линейности, перепишем последнее выражение:
$$\mathbf{b}(x,y)=\xi_1\eta_1\mathbf{b}(e_1,e_1)+\xi_1\eta_2\mathbf{b}(e_1, e_2)+\ldots\xi_n\eta_n\mathbf{b}(e_n,e_n)= \sum_{i,j}\xi_i\eta_j\mathbf{b}(e_i,e_j)=\sum_{i,j}\beta_{ij}\xi_i\eta_j.$$

Здесь $\beta_{ij}=\mathbf{b}(e_i,e_j)$ -- это значения билинейной функции на всевозможных парах базисных векторов. Их называют\textit{ коэффициентами билинейной функции} в базисе $\mathbf{e}$. Кроме того, можно составить \textit{матрицу билинейной функции}. Она имеет вид:

$$B=\begin{Vmatrix}
\beta_{11} & \beta_{12} & \cdots & \beta_{1n}\\
\beta_{21} & \beta_{22} & \cdots & \beta_{2n}\\
 \hdotsfor{4} \\
\beta_{n1} & \beta_{n2} & \cdots & \beta_{nn}
\end{Vmatrix}$$

С использованием матрицы билинейной функции и введенных обозначений, выражение для билинейной функции может быть переписано следующим образом:

$$\mathbf{b}(x,y)=\bm{\xi}^{\text{T}}B\bm{\eta}.$$

Пусть теперь помимо введенного выше базиса $\mathbf{e}$ зафиксирован еще и базис $\mathbf{e'}$. Он связан с исходным базисом через известную матрицу перехода: $\mathbf{e}'=\mathbf{e}S.$ Поставим своей задачей получить закон изменения матрицы билинейной функции при такой замене базиса. Пусть $\bm{\xi'}$ и $\bm{\eta'}$ -- координатные столбцы векторов $x$ и $y$ в базисе $\mathbf{e'}$. Тогда $\bm{\xi}=S\bm{\xi'}$ и $\bm{\eta}=S\bm{\eta'}$. Согласно выражению для билинейной функции, получим:$$\mathbf{b}(x,y)=\bm{\xi'}^{\text{T}}B'\bm{\eta'}=(S\bm{\xi'})^{\text{T}}B(S\bm{\eta'})=\bm{\xi'}^{\text{T}}(S^{\text{T}}BS)\bm{\eta'}.$$

Таким образом, в силу произвольности выбора векторов $x$ и $y$, установили, что:$$B'=S^{\text{T}}BS.$$

Важно не путать полученный результат с формулой изменения матрицы преобразования при замене базиса ($A'=S^{-1}AS$). \\

Отметим, что ранг матрицы билинейной функции не меняется при замене базиса. Действительно, умножение на невырожденную матрицу не меняет ранга исходной матрицы.\\

Билинейная функция называется симметричной, если $\forall x,y\in \mathcal{L}:$  $\mathbf{b}(x,y)=\mathbf{b}(y,x)$.
В частности, для симметричной билинейной функции:$$\mathbf{b}(e_i,e_j)=\mathbf{b}(e_j,e_i),\hspace{7pt} \forall i,j=1,\ldots n \Leftrightarrow \beta_{ij}=\beta_{ji},\hspace{7pt} \forall i,j=1,\ldots n. $$

Последнее явно указывает на то, что матрица симметричной билинейной функции симметрична. Обратное также верно.\\

\section{Квадратичные функции}
\subsection{Определение и примеры}
Квадратичной функцией на линейном пространстве $\mathcal{L}$ называется функция $\mathbf{k}$, значение которой для любого вектора $x$ из $\mathcal{L}$ определяется выражением $\mathbf{k}(x)=\mathbf{b}(x,x)$, где $\mathbf{b}$ -- симметричная билинейная функция.

Рассмотрим координатное выражение квадратичной функции (под матрицей квадратичной функции понимается матрица соответствующей билинейной функции):

$$k(x)=\bm{\xi}^{\text{T}}B\bm{\xi}=\sum_{i,j}\beta_{ij}\xi_i\xi_j.$$

Существенным отличием от билинейной функции является здесь наличие подобных членов: слагаемые $\beta_{ij}\xi_i\xi_j$ и $\beta_{ji}\xi_j\xi_i$ равны между собой, в силу симметрии. Таким образом, после приведения свободных слагаемых, квадратичная  функция принимает вид:$$\textbf{k}(x)=\beta_{11}(\xi_1)^2+2\beta_{12}\xi_1\xi_2+\beta_{22}(\xi_2)^2+2\beta_{13}\xi_1\xi_3+\ldots$$

\textbf{Пример 1.} Восстановить симметричную билинейную функцию по данной квадратичной функцию и составить ее матрицу: $\mathbf{k}(x)=x_1^2+4x_1x_2+4x_1x_3+5x_2^2+12x_2x_3+7x_3^2.$
\begin{center}
Решение:
\end{center}

Разобьем пополам перекрестные слагаемые, получим: $$\mathbf{k}(x)=\mathbf{b}(x,x)=x_1^2+2x_1x_2+2x_2x_1+2x_1x_3+2x_3x_1+
5x_2^2+6x_2x_3+6x_3x_2+7x_3^2.$$

Записанная функция является симметричной билинейной функцией. Осталось записать ее матрицу (состоит из коэффициентов при слагаемых с соответствующими индексами):
$$B=\begin{Vmatrix}
1 & 2 & 2 \\
2 & 5 & 6 \\
2 & 6 & 7
\end{Vmatrix}.$$

\textbf{Пример 2.} Записать квадратичную функцию, имеющую матрицу $B=\begin{Vmatrix}
2 & -2 & 1 \\
-2 & -1 & 0 \\
1 & 0 & 3
\end{Vmatrix}.$
\begin{center}
Решение:
\end{center}

Из матрицы прямо следует:$$\mathbf{b}(x,x)=2x_1x_1-x_2x_2+3x_3x_3-2x_1x_2-2x_2x_1+x_1x_3+x_3x_1.$$

После приведения подобных слагаемых, получим ответ:$$\mathbf{k}(x)=2x_1^2-x_2^2+3x_3^2+2x_1x_3-4x_2x_3.$$

\textbf{Пример 3.} Квадратичная форма имеет в базисе $\mathbf{e}$ вид: $\textbf{k}(x)=25x^2_1+14x_1x_2+2x^2_2.$ Выписать эту же квадратичную форму в базисе $\mathbf{e}'$, если $e'_1=e_1+e_2$, $e'_2=-e_1+e_2$.
\begin{center}
Решение:
\end{center}

Согласно условию, матрица квадратичной формы: $B=\begin{Vmatrix}
25 & 7\\
7 & 2
\end{Vmatrix}.$

Матрица переход от базиса $\mathbf{e}$ к базису $\mathbf{e}'$: $S=\begin{Vmatrix}
1 & -1\\
1 & 1
\end{Vmatrix}.$

Матрица квадратичной формы в новом базисе при этом равна:$$B'=S^{\text{T}}BS=\begin{Vmatrix}
1 & 1\\
-1 & 1
\end{Vmatrix}\cdot\begin{Vmatrix}
25 & 7\\
7 & 2
\end{Vmatrix}\cdot\begin{Vmatrix}
1 & -1\\
1 & 1
\end{Vmatrix}=\begin{Vmatrix}
41 & -23\\
-23 & 13
\end{Vmatrix}.$$

Восстановим по последней матрице квадратичную форму, получим\\

\underline{Ответ:} $k(x)=41x^2_1-46x_1x_2+13x^2_2.$\\\rule{29cm}{.20mm}

Квадратичная функция $\mathbf{k}$ в базисе $\mathbf{e}$ имеет \textbf{диагональный} вид, если в этом базисе$$\textbf{k}(x)=\sum_{i=1}^n\varepsilon_i(\xi_i)^2,$$
т.е. отсутствуют перекрестные слагаемые. Матрица квадратичной функции при этом имеет диагональный вид.

Ключевой является задача поиска базиса, в котором квадратичная функция имеет диагональный вид, а также задача нахождения выражения функции в этом базисе. Мы решим эту задачу двумя основными способами: методом элементарных преобразований и методом Лагранжа.

\subsection{Метод элементарных преобразований.}

Будем работать с матрицей $B$ квадратичной функции $\textbf{k}(x)$. Нам важно, чтобы в результате преобразований новая матрица представляла собой матрицу той же функции, но в другом базисе. Это означает, что в результате каждого шага, полученная матрица должна удовлетворять соотношению: $\widetilde B=S^{\text{T}}BS$, где $S$ - некоторая невырожденная матрица. Таким образом, например, очевидно, что одних лишь преобразований строк не достаточно (элементарное преобразование строк эквивалентно умножению на элементарную матрицу строго слева). Перед тем, как дать ответ на вопрос, как действовать дальше, вспомним ключевые вопросы касательно элементарных преобразований.

Зафиксируем некоторую матрицу $A=\begin{Vmatrix}
a & b  \\
c & d 
\end{Vmatrix}.$ 


Элементарное преобразование столбцов равносильно умножению \textbf{справа} на элементарную матрицу $S$, например:

$$AS=\begin{Vmatrix}
a & b  \\
c & d 
\end{Vmatrix} \cdot \begin{Vmatrix}
1 & 0  \\
1 & 1 
\end{Vmatrix}=\begin{Vmatrix}
a+b & b  \\
c+d & d 
\end{Vmatrix}.$$

Элементарное преобразование строк, как отмечалось, равносильно умножению \textbf{слева} на элементарную матрицу $S'$, например:

$$S'A=\begin{Vmatrix}
1 & 1  \\
0 & 1 
\end{Vmatrix} \cdot \begin{Vmatrix}
a & b  \\
c & d 
\end{Vmatrix}=\begin{Vmatrix}
a+c & b+d  \\
c & d 
\end{Vmatrix}.$$

В контексте решаемой задачи, нам необходимо, чтобы $S'=S^{\text{T}}$ (как и подобрано в примере выше). По сути это означает выполнение одной и той же операции со строками и со столбцами. То есть для того, чтобы поддерживать вид функции $\widetilde B=S^{\text{T}}BS$ необходимо совершать \textbf{сдвоенные шаги}: сначала выполнить элементарное преобразование строк, а затем сделать \textbf{то же самое} со столбцами. \\

Перейдем непосредственно к диагонализации. Итак, мы имеем дело с матрицей $$B=\begin{Vmatrix}
\beta_{11} & \beta_{12} & \cdots & \beta_{1n}\\
\beta_{21} & \beta_{22} & \cdots & \beta_{2n}\\
\hdotsfor{4}\\
\beta_{n1} & \beta_{n2} & \cdots & \beta_{nn}
\end{Vmatrix}$$ и ставим своей задачей привести ее к диагональному виду.\\

\textbf{Шаг 1.} Из каждой строчки, начиная со второй, вычтем первую строку, умноженную на коэффициент $\dfrac{\beta_{i1}}{\beta_{11}}$. Получим\footnote{Если $\beta_{11}=0$, то необходимо сделать либо перестановку $i$-той и первой строк (если нашлось $\beta_{1i}\neq0$ и $\beta_{ii}\neq0$), либо сложение этих строк (если нашлось $\beta_{1i}\neq0$ и при этом $\beta_{ii}=0$) Затем выбранное действие проделать со столбцами. Если же не нашлось $\beta_{1i}\neq0$, то переходим к следующему шагу.}:

$$\begin{Vmatrix}
\beta_{11} & \beta_{12} & \cdots & \beta_{1n}\\
0 & \cdots & \cdots & \cdots\\
\hdotsfor{4} \\
0& \cdots & \cdots & \cdots
\end{Vmatrix}$$

Теперь, в соответствии с замечаниями выше, проделываем то же самое со столбцами: из каждого столбца вычтем первый, умноженный на коэффициент $\dfrac{\beta_{1j}}{\beta_{11}}$ (в силу симметрии коэффициенты в этом и в предыдущем действии совпадают для одинаковых номеров строк и столбцов). В итоге получим:

$$ B' =\left\Vert  \begin{array}{cccc}
\beta_{11} & 0 & \ldots & 0\\
\cline{2-4}  0   & \multicolumn{1}{|c}{} &  &  \\
\vdots & \multicolumn{1}{|c}{} & C_{n-1} & \\
0 & \multicolumn{1}{|c}{} &  & 
\end{array} \right\Vert$$

В результате первого шага мы получили матрицу: $$ B'=S^{\text{T}}_{k} S^{\text{T}}_{k-1} \ldots S^{\text{T}}_1BS_1\ldots S_{k-1}S_k=S'^{\text{T}}BS'.$$

Таким образом, полученная матрица, и в частности подматрица $C_{n-1}$, также симметричные.

\textbf{Шаги 2-n.} Проделываем описанные в шаге 1 действия для подматрицы $C_{n-1}$. Итогом такого шага станет обнуление первых двух строк и столбцов, за исключением диагональных элементов. Затем повторяем действия для подматрицы $C_{n-2}$ и так далее. На последнем шаге ничего делать не придется, так как останется подматрица размера $1\times1$.\\

Итогом проделанных действий станет матрица $\widetilde B = S^{\text{T}}BS$, где $\widetilde B$ - диагональная матрица, а  $S$ - матрица перехода от исходного базиса к тому, в котором квадратичная функция имеет диагональный вид.\\

При этом легко получить и саму матрицу $S$. Она равна произведению всех элементарных матриц, задающих по ходу алгоритма преобразования \textbf{столбцов} (умножение справа). То есть $$S=S_1S_2\ldots S_{k-1}S_{k}\ldots S_p,$$ где матрицы в левой части задают элементарные преобразования \textbf{столбцов}, совершаемые по ходу алгоритма. Умножения матриц при этом легко избежать. Перепишем последнее равенство: $S=ES_1S_2\ldots S_{k-1}S_{k}\ldots S_p,$ где $E$ -- единичная матрица. Таким образом, чтобы найти матрицу перехода к базису, в котором функция имеет диагональный вид, необходимо к единичной матрице применять элементарные преобразования \textbf{столбцов}, совершенные по ходу алгоритма.\\

\textbf{Пример 4.} Привести к диагональному виду квадратичную функцию и найти матрицу перехода к базису, в котором она имеет этот вид: $\mathbf{k}(x)=2x^2_1+4x_1x_2+3x^2_2+4x_2x_3+5x^2_3.$
\begin{center}
Решение:
\end{center}

Выпишем матрицу квадратичной функции: $B=\begin{Vmatrix}
2 & 2 & 0 \\
2 & 3 & 2 \\
0 & 2 & 5
\end{Vmatrix}.$

Воспользуемся методом элементарных преобразований.

\textbf{Шаг 1.} Обнуляем все элементы первого столбца ниже первой строки. Для этого из второй строки вычтем первую. Затем то же самое проделываем со столбцами (из второго столбца вычтем первый)

$$\begin{Vmatrix}
2 & 2 & 0 \\
2 & 3 & 2 \\
0 & 2 & 5
\end{Vmatrix} \underset{\text{строки}}{\overset{(2)-(1)}{ \sim}} \begin{Vmatrix}
2 & 2 & 0 \\
0 & 1 & 2 \\
0 & 2 & 5
\end{Vmatrix} \underset{\text{столбцы}}{\overset{(2)-(1)}{ \sim}} \begin{Vmatrix}
2 & 0 & 0 \\
0 & 1 & 2 \\
0 & 2 & 5
\end{Vmatrix} $$

\textbf{Шаг 2.} Зануляем второй столбец ниже второй строки, для этого из третьей строки вычтем две вторых. Затем делаем то же со столбцами:
$$\begin{Vmatrix}
2 & 0 & 0 \\
0 & 1 & 2 \\
0 & 2 & 5
\end{Vmatrix} \underset{\text{строки}}{\overset{(3)-2(2)}{ \sim}} \begin{Vmatrix}
2 & 0 & 0 \\
0 & 1 & 2 \\
0 & 0 & 1
\end{Vmatrix} \underset{\text{столбцы}}{\overset{(3)-2(2)}{ \sim}} \begin{Vmatrix}
2 & 0 & 0 \\
0 & 1 & 0 \\
0 & 0 & 1
\end{Vmatrix}=\widetilde{B} $$

Последняя матрица диагональная. Восстановим функцию: $\mathbf{k}(\widetilde{x})=2\widetilde{x_1}^2+\widetilde{x_2}^2+\widetilde{x_3}^2$.

Для того чтобы найти матрицу перехода к новому базису, применим к единичной матрице, используемые преобразования \textbf{столбцов}:
$$\begin{Vmatrix}
1 & 0 & 0 \\
0 & 1 & 0 \\
0 & 0 & 1
\end{Vmatrix} \underset{\text{столбцы}}{\overset{(2)-(1)}{ \sim}} \begin{Vmatrix}
1 & -1 & 0 \\
0 & 1 & 0 \\
0 & 0 & 1
\end{Vmatrix} \underset{\text{столбцы}}{\overset{(3)-2(2)}{ \sim}} \begin{Vmatrix}
1 & -1 & 2 \\
0 & 1 & -2 \\
0 & 0 & 1
\end{Vmatrix} = S.$$

\underline{Ответ:} $\mathbf{k}(\widetilde{x})=2\widetilde{x_1}^2+\widetilde{x_2}^2+\widetilde{x_3}^2$, $S=\begin{Vmatrix}
1 & -1 & 2 \\
0 & 1 & -2 \\
0 & 0 & 1
\end{Vmatrix}$.\\

Диагональный вид квадратичной функции называется \textbf{каноническим}, если элементы на диагонали принимают значения из набора 1, -1 и 0.\\

\textbf{Пример 5.} Привести функцию из Примера 4 к каноническому виду.

\begin{center}
Решение:
\end{center}

Первые два шага остаются без изменений. Но возникает необходимость осуществить

\textbf{Шаг 3.} Необходимо, чтобы все диагональные элементы были из набора 1, -1 и 0. С целью привести первый элемент первой строки к единице, сделаем следующее: разделим сначала первую строку на $\sqrt{2}$, а затем то же самое сделаем с первым столбцом. Ошибкой является простое деление строки на 2, ведь за преобразованием строк \textbf{обязано} следовать преобразование столбцов. Итак:

$$\begin{Vmatrix}
2 & 0 & 0 \\
0 & 1 & 0 \\
0 & 0 & 1
\end{Vmatrix} \underset{\text{строки}}{\overset{(1):\sqrt{2}}{ \sim}} \begin{Vmatrix}
\sqrt{2} & 0 & 0 \\
0 & 1 & 0 \\
0 & 0 & 1
\end{Vmatrix} \underset{\text{столбцы}}{\overset{(1):\sqrt{2}}{ \sim}} \begin{Vmatrix}
1 & 0 & 0 \\
0 & 1 & 0 \\
0 & 0 & 1
\end{Vmatrix} =\widetilde{B}. $$

В итоге, полученный канонический вид: $\mathbf{k}(\widetilde{x})=\widetilde{x_1}^2+\widetilde{x_2}^2+\widetilde{x_3}^2$.
Матрица перехода, полученная в Примере 4, также претерпит изменения в силу шага 3. Добавится еще одно преобразование столбцов.
$$\begin{Vmatrix}
1 & -1 & 2 \\
0 & 1 & -2 \\
0 & 0 & 1
\end{Vmatrix}\underset{\text{столбцы}}{\overset{(1):\sqrt{2}}{ \sim}} \begin{Vmatrix}
\dfrac{1}{\sqrt{2}} & -1 & 2 \\
0 & 1 & -2 \\
0 & 0 & 1
\end{Vmatrix}=S$$

\underline{Ответ:} $\mathbf{k}(\widetilde{x})=\widetilde{x_1}^2+\widetilde{x_2}^2+\widetilde{x_3}^2$, $S=\begin{Vmatrix}
\dfrac{1}{\sqrt{2}} & -1 & 2 \\
0 & 1 & -2 \\
0 & 0 & 1
\end{Vmatrix}$.\\

\subsection{Метод Лагранжа.}

Метод Лагранжа заключается в последовательным выделении полных квадратов. Рассмотрим две основных ситуации.

Пусть найдется коэффициент $\beta_{ii}$ отличный от нуля. Без ограничения общности, положим $\beta_{11}\neq0$. Выделим все слагаемые, которые содержат $x_1$ и дополним их до полного квадрата. В результате вместе с полученным квадратом останется функция, зависящая только от $x_2,\ldots x_n$. С ней проделываем те же действия, и так далее. Проиллюстрируем это на примере.\\

\textbf{Пример 6.} Привести к диагональному виду квадратичную функцию и указать необходимую замену переменных: $\mathbf{k}(x)=2x^2_1+4x_1x_2+3x^2_2+4x_2x_3+5x^2_3.$

\begin{center}
Решение:
\end{center}

Выделяем все слагаемые, в которых есть $x_1$ и дополняем полученное выражение до полного квадрата:$$\mathbf{k}(x)=(2x^2_1+4x_1x_2)+3x^2_2+4x_2x_3+5x^2_3=(2x^2_1+4x_1x_2+2x_2^2)-2x_2^2+3x^2_2+4x_2x_3+5x^2_3.$$

Сворачиваем первый квадрат и приведем подобные слагаемые:$$\mathbf{k}(x)=(\sqrt{2}x_1+\sqrt{2}x_2)^2+x_2^2+4x_2x_3+5x_3^2.$$

Теперь выделим квадрат в оставшейся части:$$\mathbf{k}(x)=(\sqrt{2}x_1+\sqrt{2}x_2)^2+(x_2^2+4x_2x_3+4x_3^2)+x_3^2=(\sqrt{2}x_1+\sqrt{2}x_2)^2+(x_2+2x_3)^3+x_3^2.$$

В итоге, получаем

\underline{Ответ:} $\mathbf{k}(\widetilde{x})=\widetilde{x_1}^2+\widetilde{x_2}^2+\widetilde{x_3}^2$, где $\widetilde{x_1}=\sqrt{2}x_1+\sqrt{2}x_2$,\hspace{8pt}$\widetilde{x_2}=x_2+x_3$,\hspace{8pt}$\widetilde{x_3}=x_3$.\\

Пусть теперь не нашлось отличного от нуля коэффициента $\beta_{ii}$. Выделим какое-то перекрестное слагаемое (без ограничения общности пусть это будет $x_1x_2$) и осуществим специальную замену: $x_1=y_1+y_2$, $x_2=y_1-y_2$, $x_3=y_3, \ldots, x_n=y_n$. \\

\textbf{Пример 7.} Привести к диагональному виду квадратичную функцию и указать необходимую замену переменных: $\mathbf{k}(x)=x_2x_3-x_3x_1+x_1x_2.$

\begin{center}
Решение:
\end{center}

Осуществим специальную замену: $x_1=y_1+y_2$, $x_2=y_1-y_2$, $x_3=y_3$, получим:$$\mathbf{k}(y)=(y_1-y_2)y_3-y_3(y_1+y_2)+(y_1+y_2)(y_1-y_2)=y^2_1-y_2^2-2y_2y_3$$

Выделим полный квадрат для слагаемых, содержащих $y_2$:$$\mathbf{k}(y)=y^2_1-y_2^2-2y_2y_3=y^2_1-(y_2^2+2y_2y_3+y^2_3)+y^2_3=y^2_1-(y_2+y_3)^2+y^2_3.$$

Итого: $\mathbf{k}(\widetilde{y})=\widetilde{y_1}^2-\widetilde{y_2}^2+\widetilde{y_3}^2$, где\\

$\widetilde{y_1}=y_1=\dfrac{x_1+x_2}{2}$, $\widetilde{y_2}=y_2+y_3=\dfrac{x_1-x_2}{2}+x_3$, $\widetilde{y_3}=y_3=x_3$.\\

\underline{Ответ:} $\mathbf{k}(\widetilde{y})=\widetilde{y_1}^2-\widetilde{y_2}^2+\widetilde{y_3}^2$, где $\widetilde{y_1}=\dfrac{x_1+x_2}{2}$,\hspace{8pt}$\widetilde{y_2}=\dfrac{x_1-x_2}{2}+x_3$,\hspace{8pt}$\widetilde{y_3}=x_3$.

\subsection{Закон инерции}
\textit{Положительным индексом} ($p$) квадратичной функции называется максимальная размерность подпространства, на котором она положительно определена. Иначе говоря, это число <<$+1$>> в каноническом виде. Так, в Примере 6: $p=3$, а в Примере 7: $p=2$.\\

\textit{Отрицательным индексом} ($q$) квадратичной функции называется максимальная размерность подпространства, на котором она отрицательно определена. Иначе говоря, это число <<$-1$>> в каноническом виде. Так, в Примере 6: $q=0$, а в Примере 7: $q=1$.\\

\textit{Сигнатурой} ($s$) квадратичной функции называет разность положительного и отрицательного индекса. Так, в Примере 6: $s=3$, а в Примере 7: $s=1$.\\

\textbf{Закон инерции:} числа $p$, $q$ и $s$ не зависят от выбора базиса.

В частности, мы уже показывали, что сумма $p+q=\mathrm{Rg}B$ не зависит от выбора базиса.\\

