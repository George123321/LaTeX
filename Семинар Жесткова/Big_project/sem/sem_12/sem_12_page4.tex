\[
\begin{pmatrix*}[r]
1 & 1 & \vrule & 4 & 8 \\
1 & 3 & \vrule & 8 & 6
\end{pmatrix*}
\thicksim
\begin{pmatrix*}[r]
1 & 0 & \vrule & 2 & 9 \\
0 & 1 & \vrule & 2 & -1
\end{pmatrix*},
~ \text{где}~
A = \begin{pmatrix*}[r]
2 & 9 \\2 & -1
\end{pmatrix*}
\]
\\
В ортонормированном базисе $\Gamma = E \Rightarrow A=B$\\\\
	У симметричной квадратичной функции в ортонормированном базисе матрица $B$  равна матрице $A$. \\\\
	Если $A$ задаёт самосопряжённое преобразование $\Rightarrow$ $\exists$ ортонормированный базис из собственных векторов $\Rightarrow$ $A$ имеет диагональный вид $\Rightarrow$ $B$ также имеет диагональный вид. 
	\begin{theorem}
	В евклидовом пространстве для любой квадратичной формы существует ортонормированный базис, в котором она имеет диагональный вид.	
   \end{theorem}
\begin{prim}
	В ортонормированном базисе задана квадратичная форма. Найти ортонормированный базис, в котором она имеет диагональный вид. \\
	$k(\textbf{x}) = -4x_{1}^{2} + 10x_{1}x_{2} -4x_{2}^{2}$
\end{prim}
	$
	B =
	\begin{pmatrix*}[r]
	-4 & 5 \\
	5 & -4 
	\end{pmatrix*}
	 = A, ~\text{т.к базис ортонормированный}
	$ \\\\
	$A$ --- симметрическая $\Rightarrow$ $A$ --- самосопряжённое преобразование. \\
	$\det (A - \lambda E) = 0 \Leftrightarrow \left[
	\begin{gathered}
	\lambda_1 = 1 \\
	\lambda_2 = -9
	\end{gathered}
	\right.$
	\begin{enumerate}
		\item $\lambda =1$ \\
		$\begin{pmatrix*}[r]
			-5 & 5 & \vrule &0 \\
			5 & -5 & \vrule &0
		\end{pmatrix*};~L_1 = 
		\langle \begin{pmatrix*}[r]
		1\\ 
		1
		\end{pmatrix*}
		\rangle
		;~ \textbf{h}_1 = \cfrac{1}{\sqrt{2}}\langle \begin{pmatrix*}[r]
		1\\ 
		1
		\end{pmatrix*}
		\rangle$ 
		\item $\lambda =-9$ \\
		$\begin{pmatrix*}[r]
		5 & 5 & \vrule &0 \\
		5 & 5 & \vrule &0
		\end{pmatrix*};~L_1 = 
		\langle \begin{pmatrix*}[r]
		1\\ 
		-1
		\end{pmatrix*}
		\rangle
		;~ \textbf{h}_2 = \cfrac{1}{\sqrt{2}}\langle \begin{pmatrix*}[r]
		1\\ 
		-1
		\end{pmatrix*}
		\rangle$ 
	\end{enumerate}
В ортонормированном базисе, составленном из векторов $\textbf{h}_1$ и $\textbf{h}_2$, \\
$$A = \begin{pmatrix*}[r]
1 & 0 \\0 & -9
\end{pmatrix*} = B$$
$$k(\textbf{x}) = \tilde{x}^2_1 - 9\tilde{x}^2_2$$
\hrule
~\
\\
Вернёмся в линейное подпространство. 
\begin{theorem}Пусть даны квадратичные функции $k(\textbf{x})$ и $h(\textbf{x})$, причём $h(\textbf{x})$ --- положительно определена. Тогда в $L$ существует базис, в котором $k(\textbf{x})$ имеет диагональный вид, а $h(\textbf{x})$ -- канонический вид.
\end{theorem}
\begin{proof}
Пусть $H$ --- вспомогательное скалярное произведение, $\Gamma = H$.
\begin{enumerate}
	\item $h(\textbf{x})$ приводится к каноническому виду. В ортонормированном базисе $\hat H = E$.
	\item $K$ приводится к $\hat K$. В ортонормированном базисе $\hat K = S^{\text{T}}KS$.
\end{enumerate}
Для $\Tilde K \:\: \exists $ ОНБ, в котором она имеет диагональный вид, \\
$\Tilde K =\diag(\dots), \Tilde H =E.$\\
\end{proof}