\\
\textbf{\large{Свойства:}} \\
\hspace*{0.07cm} 1) Ортогональное преобразование инъективно.
\begin{proof}
	Пусть $\textbf{x} \in Ker~\phi \text{, т.е. } \phi(\textbf{x}) = 0. \\
	(\phi(\textbf{x}), \phi(\textbf{x})) = (\textbf{x}, \textbf{x}) = 0 \Rightarrow \textbf{x} = \textbf{o} \Rightarrow $ инъекция.
\end{proof}
\hspace*{-0.5cm}
2) Собственные значения ортогонального преобразования равны $\pm 1$.
\begin{proof}
	$ \phi (\textbf{x}) = \lambda \textbf{x}, \textbf{x} \neq \textbf{o}. \\
	(\phi(\textbf{x}), \phi(\textbf{x})) = \lambda^2(\textbf{x}, \textbf{x}) = (\textbf{x}, \textbf{x}) \Leftrightarrow \lambda^2 = 1.
	$
\end{proof}
\hspace*{-0.5cm}
3) Пусть подпространство $ U \subset \mathcal{E} $. Если $U$ инвариантно относительно $\phi$, то $U^\perp$ инвариантно относительно $\phi$.

\begin{prim}
	$\phi$ переводит столбцы матрицы $A$ в столбцы $B$. Скалярное произведение стандартное. Ортогонально ли $\phi$?
\end{prim}
\underline{Первый способ}\\
$A = \left(
\begin{array}{rr}
4 & 2 \\
7 & 1
\end{array}
\right) \rightarrow B = \left(
\begin{array}{rr}
8 & 2 \\
1 & -1
\end{array}
\right)$ \\  
\\
$
\hspace*{0.2cm} \textbf{x} \left( \begin{array}{r}
4 \\ 7
\end{array} \right) \hspace{0.7cm} \phi(\textbf{x}) \left(
\begin{array}{r}
8 \\ 1
\end{array}
\right) \hspace{0.7cm} (\textbf{x}, \textbf{y}) = 15 \\ \\
\hspace*{0.2cm} \textbf{y} \left( \begin{array}{r}
2 \\ 1
\end{array} \right) \hspace{0.7cm} \phi(\textbf{y}) \left(
\begin{array}{r}
2 \\ -1
\end{array}
\right) \hspace{0.4cm} (\phi(\textbf{x}), \phi(\textbf{y})) = 15 \\ \\ 
\begin{array}{ll}
(\textbf{x}, \textbf{x}) = 65 & (\phi(\textbf{x}), \phi(\textbf{x})) = 65 \\
(\textbf{y}, \textbf{y}) = 5 & (\phi(\textbf{y}), \phi(\textbf{y})) = 5
\end{array} \\
\hspace*{0.2cm} \Rightarrow \phi \text{ --- ортогональное.}
$ \\ 
Может возникнуть мысль, что достаточно проверить два соотношения из трех, однако этого оказывается недостаточно: \\
$ (\textbf{x}, \textbf{z}) = (\textbf{x}, \alpha \textbf{x} + \beta \textbf{y}) = \alpha \uline{(\textbf{x}, \textbf{x})} + \beta \uwave{(\textbf{x}, \textbf{y})} = (\phi(\textbf{x}), \phi(\textbf{z})) = (\phi(\textbf{x}), \alpha \phi (\textbf{x}) + \beta \phi (\textbf{y})) \\ = \alpha( \uwave{\phi(\textbf{x}), \phi(\textbf{x})}) + \beta (\uline{\phi(\textbf{x}), \phi(\textbf{y})})
$ \\
\textit{Контрпример:} \\
$
\hspace*{0.2cm} \textbf{x} \left( \begin{array}{r}
-2 \\ 2
\end{array} \right) \hspace{0.7cm} \phi(\textbf{x}) \left(
\begin{array}{r}
3 \\ 1
\end{array}
\right) \hspace{0.7cm} (\textbf{x}, \textbf{y}) = 6 \hspace*{0.5cm} (\phi(\textbf{x}), \phi(\textbf{y})) = 15 \\ \\
\hspace*{0.2cm} \textbf{y} \left( \begin{array}{r}
-2 \\ 1
\end{array} \right) \hspace{0.7cm} \phi(\textbf{y}) \left(
\begin{array}{r}
0 \\ 6
\end{array}
\right) \hspace{0.4cm} \textbf{но!} \hspace{0.2cm} (\textbf{x}, \textbf{x}) = 8 \hspace*{0.5cm} (\phi(\textbf{x}), \phi(\textbf{x})) = 10 \\ \\ 
\begin{array}{ll}
(\textbf{x}, \textbf{x}) = 65 & (\phi(\textbf{x}), \phi(\textbf{x})) = 65 \\
(\textbf{y}, \textbf{y}) = 5 & (\phi(\textbf{y}), \phi(\textbf{y})) = 5
\end{array}
$ \\ \\
Длины не сохраняются $ \Rightarrow $ не ортогонально! \vspace{0.3cm} \\
\underline{Второй способ: } \\
$$ X \left(
\begin{array}{rr}
4 & 2 \\
7 & 1
\end{array}
\right) = \left(
\begin{array}{rr}
8 & 2 \\
1 & -1
\end{array} \right) \hspace*{0.5cm}$$
Транспонируя с обеих сторон, получаем:
\vspace*{0.2cm} \\
$$\left(
\begin{array}{rr}
4 & 7 \\
2 & 1
\end{array} \right) X^{\text{T}} = \left(
\begin{array}{rr}
8 & 1 \\
2 & -1
\end{array} \right) \hspace*{0.3cm}
$$
Умножая второе уравнение на первое \textit{слева}, получим
\vspace{0.2cm} \\
$$\left(
\begin{array}{rr}
4 & 7 \\
2 & 1
\end{array} \right) X X^{\text{T}} \left(
\begin{array}{rr}
4 & 2 \\
7 & 1
\end{array} \right) = \left(
\begin{array}{rr}
8 & 1 \\
2 & -1
\end{array} \right) \left(
\begin{array}{rr}
8 & 2 \\
1 & -1
\end{array} \right)
$$