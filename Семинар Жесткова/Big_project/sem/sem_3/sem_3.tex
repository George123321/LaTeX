\chapter{Линейные пространства и подпространства.}
\section{Определение линейного пространства}
\begin{definition} % definition
	Пространство $L$ ~---~ \underline{линейное пространство}, если:
	\begin{itemize}
		\item $\forall x, y \in L: x + y \in L$
		\item $\forall x \in L, \forall \alpha \in \mathbb{R} : \alpha x \in L$
	\end{itemize}
	
	+ 8 аксиом:
	\begin{enumerate}
		\item $x + y = y+x$
		\item $(x+y) +z=x +(y+z)$
		\item $\exists\ o : \forall x \rightarrow x+o=x$
		\item $\exists\ (-x):\forall x \rightarrow  x+(-x) = o$
		\item $\alpha(x+y) = \alpha x+\alpha y$
		\item $(\alpha +\beta)x = \alpha x+\beta x$
		\item $(\alpha\beta) x=\alpha(\beta x)$
		\item $\exists\ 1: x\cdot 1= x$
	\end{enumerate}
\end{definition} 
\underline{Вектор}~---~ элемент линейного пространства.

\begin{itemize}
	\item Понятия ЛЗ и ЛНЗ со всеми вытекающими свойствами полностью из аналита.
\end{itemize}

\begin{definition}
	\underline{Базис в $L$}~---~ конечная, упорядоченная ЛНЗ система векторов, такая что каждый вектор из $L$ по ней раскладывается.
\end{definition}
Если базис состоит из $n$ векторов, то пространство называется \underline{$n$-мерным} ($\dim L = n$). % использование \dim для функций (в литературе принято писать функции прямыми буквами)
\vspace{3mm}

Примеры:
\begin{itemize}
	\item Векторы в $3^x$ $(\dim L = n)$. Базис: $\left\{\left(\begin{array}{c} % зачем создавать по 2 столба? Можно ограничиться окружением \begin{pmatrix*}[r]
	1\\  
	0\\
	0\\
	\end{array}\right),\left(\begin{array}{c}
	0\\  
	1\\
	0\\
	\end{array}\right),\left(\begin{array}{c}
	0\\  
	0\\
	1\\
	\end{array}\right)\right\}$
	\item Столбцы высотой $n$ $(\dim L = n)$. Базис: $\left\{\left(\begin{array}{c}
	1\\  
	0\\
	\vdots\\
	0\\
	\end{array}\right),\ldots,\left(\begin{array}{c}
	0\\
	0\\
	\vdots\\
	1\\
	\end{array}\right)\right\}$
	\item Матрицы $m\times n$ $(\dim L =m\times n)$. Базис: $\left\{\begin{pmatrix*}[r]
	1 & 0 & \cdots & 0 \\  
	0 & 0 & \cdots & 0 \\    
	\hdotsfor{4} \\
	0 & 0 & \cdots & 0
	\end{pmatrix*},\begin{pmatrix*}[r]
	0 & 1 & \cdots & 0 \\  
	0 & 0 & \cdots & 0 \\       
	\hdotsfor{4} \\
	0 & 0 & \cdots & 0
	\end{pmatrix*},\dots,\begin{pmatrix*}[r]
	0 & 0 & \cdots & 0 \\  
	0 & 0 & \cdots & 0 \\       
	\hdotsfor{4} \\
	0 & 0 & \cdots & 1
	\end{pmatrix*}\right\}$
	\item Множество функций, определённых на отрезке $[0,1]$ % избегаем сокращений.
	\item Многочлены $(\dim L = \infty)$
	\item Многочлены степени $\leq n$ $(\dim L = n+1)$.~Базис:$\{1, t, t^2, \dots,  t^n\}$ %после запятых все-таки нужны пробелы
\end{itemize}

\begin{definition}
	\underline{Линейное подпространство}.~$L'$~---~ линейное подпространство в $L$, если:
	\begin{itemize}
		\item $\forall x, y \in L': x + y \in L'$ % пробел между x, y
		\item $\forall x \in L', \forall \alpha \in \mathbb{R} : \alpha x \in L'$
	\end{itemize}
\end{definition}

Пример: диагональные матрицы в пространстве обычных матриц.
\section{Примеры}
В примерах 1~---~3 вопрос следующий: является ли данное множество линейным подпространством в данном пространстве $L$.
\begin{prim} % я использую специальное окружение prim, чтобы не париться с нумерацией
	L --- множество n-мерных векторов. % в литературе тире записывается как ---
\end{prim}
a) L' --- множество векторов, координаты которых равны\\
Да, является;  $\dim L'=1$, базис: % использование \dim для функций (в литературе принято писать функции прямыми буквами) 
$\left\{
\begin{pmatrix*}[c] % для удобства чтения проще размещать строки друг под другом
1\\ 
1\\ 
\vdots\\
1\\ 
\end{pmatrix*}
\right\}$\\
б) L' --- множество векторов,  сумма координат которых равна 0\\
Да, является;  $\dim  L'=n-1$, базис: 
$
\left\{ % Можно использовать \left\{ для того, чтобы скобка подгонялась под нужный размер 
\begin{pmatrix*}[c] % окружение smallmatrix почему-то приводит к косякам вроде лишнего пробела перед многоточием. Думаю, проще использовать просто pmatrix, тем более, что место неограничено
1\\ 
0\\ 
\vdots\\ 
0\\ 
-1\\
\end{pmatrix*}
, \cdots ,
\begin{pmatrix*}[c]
0\\ 
\vdots\\ 
0\\ 
1\\ 
-1\\
\end{pmatrix*}
\right\}$\\
в) L' --- множество векторов,  сумма координат которых равна 1\\
Нет, не является.\\

\begin{prim}
	L --- множество матриц размера $n \times n$.
\end{prim}
a) L' --- матрицы с нулевой первой строкой\\
Да, является;  $\dim  L'=n^2-n$\\
б) L' --- множество диагональных матриц\\
Да, является;  $\dim  L'=n$\\
в) L' --- множество верхнетреугольных матриц\\
Да, является;  $\dim  L'=\frac{n(n+1)}{2}$   (т.е. $(1+2+\cdots +n)$) \\
г) L' --- множество вырожденных матриц\\
Нет, не является;
$\left( % здесь действительно лучше использовать smallmatrix
\begin{smallmatrix}
0&0 \\
0&1 
\end{smallmatrix} 
\right)$  
$+$
$\left(
\begin{smallmatrix}
1&0 \\
0&0 
\end{smallmatrix} 
\right)$  
$=$
$\left(
\begin{smallmatrix}
1&0 \\
0&1 
\end{smallmatrix} 
\right)$  
\\

\begin{prim}
	L --- множество функций, определенных на отрезке [0,1].
\end{prim}
a) L' --- множество функций, ограниченных на отрезке [0,1]\\
Да, является. \\
б) L' --- множество строго монотонных функций\\
Нет, не является. \\
в) L' --- множество строго возрастающих функций\\
$0\cdot x = 0 \Longrightarrow$ нет, не является. \\

\section{Примеры и способы задания линейных подпространств}

0 --- тоже линейное пространство
\begin{definition}
	Линейная оболочка векторов $a_1, a_2, \cdots , a_k$  $(\langle a_1, a_2, \cdots , a_k\rangle )$ --- всевозможные линейные комбинации этих векторов:
	$$\langle a_1, a_2, \cdots , a_k\rangle =\left\{ \sum\limits^k _{i= 1} \lambda_i  a_i  , \lambda_i \in R \right\}$$ 
\end{definition}

\begin{prim}
	Найти размерность и базис линейной оболочки
\end{prim}
$$
\langle 
\begin{pmatrix*}[r]
1\\ 
1\\ 
1\\ 
1\\
\end{pmatrix*}
,
\begin{pmatrix*}[r]
1\\ 
1\\ 
1\\ 
3\\
\end{pmatrix*}
,
\begin{pmatrix*}[r]
3\\ 
-5\\ 
7\\ 
2\\
\end{pmatrix*}
,
\begin{pmatrix*}[r]
1\\ 
-7\\ 
5\\ 
2\\
\end{pmatrix*}
\rangle 
$$
Т.к. $\dim L'=\Rg A$:
$$
A=
\begin{pmatrix*}[r]
1 & 1 & 1 & 1 \\
1 & 1 & 1 & 3 \\
3 & -5 & 7 & 2 \\
1 & -7 & 5 & 2 \\ 
\end{pmatrix*} 
\xrightarrow[(3)-3(1)]{ % стрелка позволят писать и под ней
	(2)-(1),\\
	(4)-(1)}
\begin{pmatrix*}[r]
1 & 1 & 1 & 1 \\
0 & 0 & 0 & 2 \\
0 & -8 & 4 & 1 \\
0 & -8 & 4 & 1 \\ 
\end{pmatrix*} \\
$$
$(3)=(4) \Rightarrow (4)$ вычеркиваем!\\
$\dim  L'=3$, базис: $
\left\{
\begin{pmatrix*}[r]
1\\ 
1\\ 
1\\ 
1\\
\end{pmatrix*}
,
\begin{pmatrix*}[r]
0\\ 
0\\ 
0\\ 
2\\
\end{pmatrix*}
,
\begin{pmatrix*}[r]
0\\ 
-8\\ 
4\\ 
1\\
\end{pmatrix*}
\right\}$

\begin{prim}
	(условие --- см. пример 4)
\end{prim}
$$\langle 
\begin{pmatrix*}[r]
6 & 8 & 9 \\
0 & 1 & 6 \\
\end{pmatrix*} 
,
\begin{pmatrix*}[r]
2 & 1 & 1 \\
3 & 0 & 1 \\
\end{pmatrix*} 
,
\begin{pmatrix*} % из-за выравнивания по правому краю, появляется много лишнего места. имхо, это не очень красиво
2 & 6 & 7 \\
-6 & 1 & 4 \\
\end{pmatrix*} 
\rangle $$
Задача не изменится, если взять
$$\langle 
\begin{pmatrix*}[r]
6 \\ 
8 \\ 
9 \\
0 \\ 
1 \\ 
6 \\
\end{pmatrix*} 
,
\begin{pmatrix*}[r]
2 \\ 
1 \\ 
1 \\
3 \\ 
0 \\ 
1 \\
\end{pmatrix*} 
,
\begin{pmatrix*}
2 \\ 
6 \\ 
7 \\
-6 \\ 
1 \\ 
4 \\
\end{pmatrix*} 
\rangle  \\
\begin{pmatrix*}
6 & 8 & 9 & 0 & 1 & 6 \\
2 & 1 & 1 & 3 & 0 & 1 \\
2 & 6 & 7 & -6 & 1 & 4 \\
\end{pmatrix*} 
$$
Т.к. строка (3) ЛНЗ ((1)-2(2)=(3)), ее можно вычеркнуть.\\
$\dim  L'=2$, базис: 
$\left\{
\begin{pmatrix*}[r]
6 & 8 & 9 \\
0 & 1 & 6 \\
\end{pmatrix*} 
,
\begin{pmatrix*}[r]
2 & 1 & 1 \\
3 & 0 & 1 \\
\end{pmatrix*} 
\right\}$\\
\begin{prim}
	Доказать, что матрицы A, B, C, D образуют базис в пространстве матриц $2 \times 2$ и найти координаты вектора F в этом базисе.
\end{prim}
$$
A=
\begin{pmatrix*}
1 & -1 \\
1 & 1  \\
\end{pmatrix*} 
,\  % в мат. режиме для пробелов можно использовать \ 
B=
\begin{pmatrix*}[r]
2 & 5 \\
1 & 3  \\
\end{pmatrix*}
,\  
C=
\begin{pmatrix*}[r]
1 & 1 \\
0 & 1  \\
\end{pmatrix*} 
, \ 
D=
\begin{pmatrix*}[r]
3 & 4 \\
5 & 7  \\
\end{pmatrix*} 
,\  
F=
\begin{pmatrix*}[r]
5 & 14 \\
6 & 13  \\
\end{pmatrix*} 
$$
Если $A, B, C, D$ --- базис, то $\exists !\ x_1, x_2, x_3, x_4$:
$$
Ax_1+Bx_2+Cx_3+Dx_4=F,
$$
что эквивалентно СЛУ:
$$
\left\{
\begin{array}{rrrrl}
x_1&+2x_2&+x_3&+3x_4&=5\\
-x_1&+5x_2&+x_3&+4x_4&=14\\
x_1&+x_2&&+5x_4&=6\\
x_1&+3x_2&+x_3&+7x_4&=13\\
\end{array}
\right.
$$
Решая эту СЛУ, получим:
$$
\begin{pmatrix*}[r]
x_1\\
x_2\\
x_3\\
x_4\\
\end{pmatrix*}
=
\begin{pmatrix*}[r]
-1\\
2\\
-1\\
1\\
\end{pmatrix*}
$$
Т.о., решив систему, убедились в единственности решения (факт базиса).
\begin{prim}
	Найти размерность и базис линейной оболочки.
\end{prim}
$$
\langle (1+t)^3, t^3, t+t^2, 1\rangle 
$$
Стандартный базис многочлена: $\left\{1, t, t^2, t^3 \right\}$.
Тогда координаты наших векторов в стандартном базисе есть:
$$
\langle 
\begin{pmatrix*}[r]
1\\
3\\
3\\
1\\
\end{pmatrix*}
,
\begin{pmatrix*}[r]
0\\
0\\
0\\
1\\
\end{pmatrix*}
,
\begin{pmatrix*}[r]
0\\
1\\
1\\
0\\
\end{pmatrix*}
,
\begin{pmatrix*}[r]
1\\
0\\
0\\
0\\
\end{pmatrix*}
\rangle 
$$
Тогда запишем матрицу, аналогично примеру 4:
$$
\begin{pmatrix*}[r]
1 & 3 & 3 & 1\\
0 & 0 & 0 & 1\\
0 & 1 & 1 & 0\\
1 & 0 & 0 & 0\\
\end{pmatrix*}
$$
1 строка ЛНЗ, ее можно вычеркнуть. Тогда
$$
\dim L'=3, \text{ базис: } \left\{t^3, t^2+t, 1\right\}.
$$
\begin{prim}
	Доказать, что
	$$
	1, t-\alpha, (t-\alpha)^2, \dots, (t-\alpha)^n
	$$
	--- базис в пространстве многочленов, степени не выше $n$. Найти в этом базисе разложение $P_n(t)$.
\end{prim}
Ответ на эту задачу дал математик Брук Тейлор:
$$
P_n(t)=P_n(\alpha)+\frac{1}{1!}P'(\alpha)(t-\alpha)+\dots+\frac{1}{n!}P^{(n)}_n(\alpha)(t-\alpha)^n
$$
Т.к. данное разложение $\exists !$, это базис. Запишем коэффициенты разложения:
$$
\begin{pmatrix*}[c]
P_n(\alpha)\\
\frac{1}{1!}P'_n(\alpha)\\
\hdotsfor{1}\\
\frac{1}{n!}P^{(n)}_n(\alpha)
\end{pmatrix*}
$$
\begin{prim}
	Найти размерность и базис подпространства, заданного в виде $Ax=0$, где
	$$
	A = 
	\begin{pmatrix*}[r]
	1 & 2 & 0 & 1\\
	3 & 4 & -2 & 5\\
	\end{pmatrix*}.
	$$
\end{prim}
Решения однородной системы образуют линейные подпространства.\\
$
\left(
\begin{array}{cccc|c}
1 & 2 & 0 & 1 & 0\\
3 & 4 & -2 & 5 & 0\\
\end{array}
\right)
\xrightarrow{\text{Алгоритм Гаусса}}
\left(
\begin{array}{cccc|c}
1 & 0 & -2 & 3 & 0\\
0 & 1 & 1 & -1 & 0\\
\end{array}
\right)
$\\
$
\begin{pmatrix*}[r]
x_1\\
x_2\\
x_3\\
x_4\\
\end{pmatrix*}
=
\begin{pmatrix*}[r]
2 & -3\\
-1 & 1\\
1 & 0\\
0 & 1\\
\end{pmatrix*}
\begin{pmatrix*}[r]
c_1\\
c_2\\
\end{pmatrix*}
$\\
Отсюда, получаем линейную оболочку:
$$
\langle 
\begin{pmatrix*}[r]
2\\
-1\\
1\\
0\\
\end{pmatrix*}
,
\begin{pmatrix*}[r]
-3\\
1\\
0\\
1\\
\end{pmatrix*}
\rangle 
, \qquad \dim L' = 2
$$
Столбцы этой линейной оболочки будут являться базисом в этом линейном подпространстве.
\begin{prim}
	Задать подпространство в виде однородной системы
	$$
	\langle 
	\begin{pmatrix*}[r]
	1\\
	2\\
	3\\
	4\\
	\end{pmatrix*}
	,
	\begin{pmatrix*}[r]
	0\\
	1\\
	-1\\
	1\\
	\end{pmatrix*}
	\rangle .
	$$
\end{prim}
Задача по сути является задачей из прошлого семинара:\\
$
\left(
\begin{array}{rr|r}
1 & 0 & x_1\\
2 & 1 & x_2\\
3 & -1 & x_3\\
4 & 1 & x_4\\
\end{array}
\right)
\rightarrow
\left(
\begin{array}{cc|c}
1 & 0 & x_1\\
0 & 1 & x_2-2x_1\\
0 & 0 & -5x_1+x_2+x_3\\
0 & 0 & -2x_2-x_2+x_4\\
\end{array}
\right)
$\\
Для того, чтобы система была совместной, требуется равенство 0 последних двух строк. Отсюда ответ:
$$
\left\{
\begin{array}{rrrrl}
-5x_1&+x_2&+x_3&&=0\\
-2x_1&-x_2&&+x_4&=0\\
\end{array}
\right.
$$