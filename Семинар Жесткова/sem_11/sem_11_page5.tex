Этот базис можно сделать ортогональным в силу леммы \ref{lemm2}.

\textbf{Геометрический смысл}\\
1) "Сжатие"\ вдоль перпендикулярного направления\\
2) Ортогональное проецирование\\
3) Отражение
\begin{prim}
В ортонормированном базисе $\varphi$ задана матрица $ A=
\left(
\begin{matrix*}[r]
1 & 2\\ 
2 & -2\\
\end{matrix*}
\right). 
$
Найти ОНБ из собственных векторов.
\end{prim}
В ОНБ: $-A=A^{\textbf{T}} \Rightarrow \varphi$ --- самосопряженное преобразование.\\
$$
\det (A-\lambda E)=0 \Leftrightarrow \lambda^2 +\lambda -6=0  \Leftrightarrow
\left[
\begin{aligned}
\lambda=-3\\
\lambda=-2\\
\end{aligned}
\right.
$$
$$
1) \lambda=-3 \:\:\: 
\begin{pmatrix*}[r]
 4 & 2 & \vrule & 0\\
 2 & 1 & \vrule & 0\\
\end{pmatrix*}
\:\:\:  
L_1:
 \langle \left(
\begin{matrix*}[r]
-1\\ 2\\ 
\end{matrix*}
\right) \rangle
=\textbf{f}_1
$$
$$
1) \lambda=2 \:\:\: 
\begin{pmatrix*}[r]
 -1 & 2 & \vrule & 0\\
 2 & -4 & \vrule & 0\\
\end{pmatrix*}
\:\:\:  
L_1:
 \langle \left(
\begin{matrix*}[r]
2\\ 1\\ 
\end{matrix*}
\right) \rangle
=\textbf{f}_2
$$
$$
\left.
\begin{aligned}
\textbf{e'}_1 =\cfrac{1}{\sqrt{5}}
\begin{pmatrix*}[r]
-1\\ 2\\ 
\end{pmatrix*}
 \\
\textbf{e'}_2 =\cfrac{1}{\sqrt{5}}
\begin{pmatrix*}[r]
2\\ 1\\ 
\end{pmatrix*}
 \\
\end{aligned}
\right.
 \:\:\:\:\:\: A'=
\begin{pmatrix*}[r]
 -3 & 0 \\
 0 & 2 \\
\end{pmatrix*}
$$
\begin{prim}
	В ортонормированном базисе $\varphi$ задана матрица $A$. Найти ОНБ из собственных векторов.
$$
A=
\begin{pmatrix*}[r]
 1 & 2 & 2 \\
 2 & 1 & -2 \\
 2 & -2 & 1 \\
\end{pmatrix*}
 \:\:\:\:\:\:
\text{В ОНБ} \:\:\: A=A^{\textbf{T}}  \Rightarrow \varphi \text{ --- самосопряженное}
$$
\end{prim}
$$
\det (A-\lambda E)=0 \Leftrightarrow (\lambda-3)^2 (\lambda +3)=0  \Leftrightarrow
\left.
\begin{aligned}
\lambda=3 \:\:\:  \text{кратность 2}\\
\lambda=-3  \:\:\: \text{кратность 1}\\
\end{aligned}
\right.
$$

$$
1) \lambda=3 \:\:\: 
\begin{pmatrix*}[r]
 -2 & 2 & 2 & \vrule & 0\\
 2 & -2 & -2 & \vrule & 0\\
 2 & -2 & -2 & \vrule & 0\\
\end{pmatrix*}
\:\:\:  
L_1:
 \langle \left(
\begin{matrix*}[r]
1\\ 1\\ 0\\ 
\end{matrix*}
\right)=\textbf{f}_1,
\left(
\begin{matrix*}[r]
1\\ 0\\ 1\\ 
\end{matrix*}
\right)=\textbf{f}_2 \rangle
$$

$$
2) \lambda=-3 \:\:\: 
\begin{pmatrix*}[r]
 4 & 2 & 2 & \vrule & 0\\
 2 & 4 & -2 & \vrule & 0\\
 2 & -2 & 4 & \vrule & 0\\
\end{pmatrix*}
\:\:\:  
L_2:
 \langle \left(
\begin{matrix*}[r]
-1\\ 1\\ 1\\ 
\end{matrix*}
\right)=\textbf{f}_2 \rangle
$$
$
\textbf{h}_1=\textbf{f}_1\\
\textbf{h}_2=\textbf{f}_2 - \cfrac{(\textbf{f}_2,\textbf{h}_1)}{|\textbf{h}_1|^2} \textbf{h}_1=\left(
\begin{matrix*}[r]
1\\ -1\\ 2\\ 
\end{matrix*}
\right)\\
\textbf{h}_3=\textbf{f}_3\\
\textbf{e}_1 =\cfrac{1}{\sqrt{2}}
\begin{pmatrix*}[r]
1\\ 1\\ 0\\ 
\end{pmatrix*}
\textbf{e}_2 =\cfrac{1}{\sqrt{6}}
\begin{pmatrix*}[r]
1\\ -1\\ 2\\  
\end{pmatrix*}
\textbf{e}_3 =\cfrac{1}{\sqrt{3}}
\begin{pmatrix*}[r]
-1\\ 1\\ 1\\  
\end{pmatrix*}
 \\
$
В этом базисе:
$$
A = \begin{pmatrixr}
3&0&0\\
0&3&0\\
0&0&-3\\
\end{pmatrixr}
$$