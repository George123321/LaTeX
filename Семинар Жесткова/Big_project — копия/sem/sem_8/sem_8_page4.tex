$$A_{\phi}=\begin{pmatrix}
\cos \alpha & -\sin \alpha & 0\\
\sin \alpha & \cos \alpha & 0\\
0 & 0 & 1\\
\end{pmatrix}$$
Т.к.
$\tr A=2\cos \alpha +1 = \const:$
$$\boxed{\cos \alpha = \cfrac{1}{2}~ (\tr A-1)}$$

\begin{prim}	
Пусть $\lambda_1,~ \dots ~, \lambda_n$ --- собственные значения преобразования $\varphi$ с матрицей A. Какие собственные значения у ~а) $\varphi^2$;~ б)$\varphi^{-1}$ ?
	\end{prim}
\begin{align}
	\det(A-\lambda E)=0  \tag{$*$}
	\label{har} 
\end{align}

\begin{enumerate}
	\item[1).] $\varphi^2:~\det(A^2-\widetilde \lambda E)$\\
	$$~\eqref{har} |\cdot \det(A+\lambda E) \Rightarrow \det(A-\lambda E) \det(A+\lambda E)=\det(A^2-\lambda^2 E) = 0~~\Rightarrow \widetilde \lambda = \lambda^2$$
	
	\item[2).] $\varphi^{-1}:~\det(A^{-1}-\tilde {\lambda} E)= \det(A^{-1}-\tilde { \lambda} A^{-1}A)=0$%две тильды - ужс
	$$\Rightarrow \det A^{-1} \cdot \det(E-\tilde {\lambda} A)=0$$
	Так как $\det A^{-1} \ne 0$ (матрица $A$ невырожденная), то верно:
	 $$\det(E-\tilde {\lambda} A)=0$$
	 Разделим равенство на $(-1)^n \tilde {\lambda}^n$:
	 $$\det(A-\cfrac{E}{\tilde {\lambda}})=0 ~\Rightarrow~ \tilde {\lambda}=\cfrac{1}{\lambda}=\lambda^{-1}.$$
	 Заметим, что собственные векторы при этом не изменятся.
\end{enumerate}

\section{Проекторы}
Пусть $L=L_1 \oplus L_2$;\\
$\forall \textbf{x}\in L: \textbf{x}=\textbf{x$_1$}+\textbf{x$_2$}$ --- единственный прообраз, где $\textbf{x$_1$} \in L_1,~ \textbf{x$_2$} \in L_2$
\begin{center}
Тогда: $P_1(\textbf{x})=\textbf{x$_1$}$ --- проектор на $L_1 \parallel L_2$\\
$P_2(\textbf{x})=\textbf{x$_2$}$ --- проектор на $L_2 \parallel L_1$\\
$\im P_1 = L_1$ и $\ker P_1=L_2$\\
$\im P_2 = L_2$ и $\ker P_2=L_1$\\
\end{center}
$$P_1(\textbf{x})+P_2(\textbf{x})=\textbf{x$_1$}+\textbf{x$_2$}=\textbf{x}~\Rightarrow ~(P_1+P_2)(\textbf{x})=\textbf{x}~ \Rightarrow ~P_1+P_2=\Id$$%преобразование \Id