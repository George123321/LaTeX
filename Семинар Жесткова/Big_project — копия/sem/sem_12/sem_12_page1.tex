\section{Ортогональные матрицы.}
Рассмотрим два ОНБ
$$
\textbf{e} \text{ и } \textbf{e}'=\textbf{e}S.
$$
В этих базисах
$$
\Gamma = E \text{ и } \Gamma' = E.
$$
Мы знаем, что
$$
\Gamma'=S^{\text{T}}\Gamma S.
$$
Из этих равенств следует, что
\begin{equation}
S^{\text{T}}S=E \Rightarrow S^{\text{T}}=S^{-1}.
\tag{$*$}
\label{ortog}
\end{equation}
Такие матрицы, для которых выполнено \eqref{ortog}, называются ортогональными, причем
$$
\det{S^{\text{T}}S}=\det{S}\det{S^{\text{T}}}=\det{E}.
$$
Т.к. $\det{S}=\det{S^{\text{T}}}$, то
$$
\Det^2S=1 \Rightarrow \det{S} = \pm 1.
$$

Рассмотрим подробнее матрицу $S$.\\
Матрица $S$ состоит из столбцов $s_i^{\uparrow}$
$$
S = \begin{pmatrix}
s_1^{\uparrow}&s_2^{\uparrow}&\cdots&s_n^{\uparrow}
\end{pmatrix},
$$
тогда $S^{\text{T}}$ из строк $\vec{s}_i$
$$
S^{\text{T}} = \begin{pmatrix}
\vec{s}_1\\\vec{s}_2\\\vdots\\\vec{s}_n
\end{pmatrix}.
$$
Т.к. для матрицы $S$ выполнено \eqref{ortog}, то
$$
\begin{pmatrix}
s_1^{\uparrow}&s_2^{\uparrow}&\cdots&s_n^{\uparrow}
\end{pmatrix}
\begin{pmatrix}
\vec{s}_1\\\vec{s}_2\\\vdots\\\vec{s}_n
\end{pmatrix}
=
E,
$$
откуда следует, что
$$
\begin{cases}
s_is_j=1, \forall i=j\\
s_is_j=0, \forall i\neq j\\
\end{cases}
\Leftrightarrow
s_is_j=\delta_{ij},
$$
где $\delta_{ij}$ --- символ Кронекера.

Таким образом, столбцы/строки матрицы $S$ формируют ОНБ.

Если мы имеем дело с матрицами размерами $2\times 2$, то они имеют вид:
$$
S = \begin{pmatrixr}
\cos\alpha&\mp\sin\alpha\\
\sin\alpha&\pm\cos\alpha\\
\end{pmatrixr}.
$$
\section{Ортогональное преобразование.}
\begin{definition}
Преобразование $\phi$ с матрицей $A$ евклидового пространства $\mathcal{E}$ называется ортогональным, если оно сохранет скалярное произведение, т.е.
$$
\forall \textbf{x}, \textbf{y}\in\mathcal{E}\longmapsto(\phi(\textbf{x}), \phi(\textbf{y})) = (\textbf{x}, \textbf{y}).
$$
\end{definition}
\noindent Также сохраняются углы и длины --- геометрический смысл <<движения>>.\\

Из семинара 11:
$$
\forall \textbf{x}, \textbf{y} \in \mathcal{E} \longmapsto (\phi(\textbf{x}), \textbf{y}) = (\textbf{x}, \phi^*(\textbf{y})),
$$
где $\phi^*$ --- сопряженное преобразование. Заменим $\textbf{y}$ на $\phi(\textbf{y})$ (т.к. говорим об ортогональных преобразованиях)
$$
(\textbf{x}, \textbf{y})=(\phi(\textbf{x}), \phi(\textbf{y}))=(\textbf{x}, \phi^*\phi(\textbf{y})).
$$
Используя свойство линейности, перепишем равенство так
$$
(\textbf{x}, \phi^*\phi(\textbf{y})-\textbf{y})=0.
$$
Т.к. это равенство выполнено для любых \textbf{x}, \textbf{y}:
$$
\phi^*\phi(\textbf{y})-\textbf{y}=\textbf{o}\Rightarrow\phi^*\phi(\textbf{y})=\textbf{y}\Rightarrow\phi^*\phi=\Id,
$$
где $\Id$ --- тождественное преобразование. Т.е. мы получили, что $A^*A=E$.
В ОНБ $A^*=A^{\text{T}}$, откуда следует, что
$$
\boxed{A^{\text{T}}A=E},
$$
т.е. ортогональное преобразование задает ортогональная матрица.