Далее подставим числа $\lambda$ в $(*)$:

\begin{enumerate}
\item{
$\lambda_1=1 :$
$\begin{pmatrix*}[r]
 1 & 2 & 1 & \vrule & 0\\
 -2 & -4 & 2 & \vrule & 0\\
 3 & 6 & -1 & \vrule & 0\\
\end{pmatrix*} , $\\
$
L_1=
\left(
\begin{smallmatrix*}[r]
x_1\\ x_2\\ x_3\\ 
\end{smallmatrix*}
\right) 
=
\langle
\left(
\begin{smallmatrix*}[r]
2\\ -1\\ 0\\ 
\end{smallmatrix*}
\right) 
\rangle
$
$\leftarrow $
 Собственный вектор, перейдёт сам в себя т.к. $\lambda=1$.
}
\item{
$\lambda_2=3 :
L_2=
\left(
\begin{smallmatrix*}[r]
x_1\\ x_2\\ x_3\\ 
\end{smallmatrix*}
\right) 
=
\langle
\left(
\begin{smallmatrix*}[r]
1\\ 0\\ 1\\ 
\end{smallmatrix*}
\right) 
\rangle
$
}
\item{
$\lambda_3=-5 :
L_3=
\left(
\begin{smallmatrix*}[r]
x_1\\ x_2\\ x_3\\ 
\end{smallmatrix*}
\right) 
=
\langle
\left(
\begin{smallmatrix*}[r]
1\\ -8\\ 9\\ 
\end{smallmatrix*}
\right) 
\rangle
$
}
\end{enumerate}

\begin{lemma} % окружение лемма!
	Собственные векторы, соответствующие различным собственным значениям, попарно линейно независимы.
\end{lemma}


Соберём базис $\textbf{f}$ %f в мат режиме
$  %для красоты пришлось сделать ужас: в матрице матрица, у которой в подписи матрица.
\begin{matrix}
\overset{\left\{
\begin{pmatrix*}[r]
2\\ -1\\ 0\\ 
\end{pmatrix*},
\begin{pmatrix*}[r]
1\\ 0\\ 1\\ 
\end{pmatrix*},
\begin{pmatrix*}[r]
1\\ -8\\ 9\\ 
\end{pmatrix*}
\right\}}{
\begin{matrix*}[r]
\textbf{f$_1$}\phantom{3}&\textbf{f$_2$}&\phantom{2}\textbf{f$_3$}
\end{matrix*}}
\end{matrix}
$\\
$
\left.
\begin{aligned} %раз уж aligned, сделай выравнивание
\varphi (\textbf{f$_1$})&=\textbf{f$_1$}\\ %ВЕКТОРЫ
\varphi (\textbf{f$_2$})&=3\textbf{f$_2$} \\
\varphi (\textbf{f$_3$})&=5\textbf{f$_3$}\\
\end{aligned}
\right.
\to A'=
\begin{pmatrix*}[r]
 1 & 0 & 0\\
 0 & 3 & 0\\
 0 & 0 & -5\\
\end{pmatrix*}
$
в базисе $\textbf{f}$.\\ %точки в конце предложений!

\section{Диагонализируемость матрицы}
\begin{prim}
Диагонализировать матрицу\\
$$
 A=
\begin{pmatrix*}[r]
 3 & 1 & -2\\
 2 & 2 & -2\\
 2 & 1 & -1\\
\end{pmatrix*}
$$
\end{prim}\\

$
\det(A-\lambda E)=0$:\\ %заебался уже говорить о мат функциях, об их оформлениях. Почему нельзя делать правильно - я не понимаю.
$
\begin{vmatrix*}[c] %лучше сделать центрирование
 3-\lambda & 1 & -2\\
 2 & 2-\lambda & -2\\
 2 & 1 & -1-\lambda\\
\end{vmatrix*}
=0
\leftrightarrow
	\underset{ \text{Не забывайте про свойства детерминанта}}{(\lambda-1)^2(\lambda-2)=0}\\
$
$\lambda =1$ --- корень алгебраической кратности 2.\\ % тире тройное!!!!!!!!!!!!+что такое алгоритмическая кратность?
$\lambda =2$ --- корень алгебраической кратности 1 (простой корень).\\


